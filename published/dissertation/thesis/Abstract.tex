%Abstract Page

\hbox{\ }

\renewcommand{\baselinestretch}{1}
\small \normalsize

\begin{center}
\large{{ABSTRACT}}

\vspace{3em}

\end{center}
\hspace{-.15in}
\begin{tabular}{ll}
Title of dissertation:    & {\large  PLANETARY SCALE DATA STORAGE }\\
\ \\
&                          {\large  Benjamin Bengfort} \\
&                           {\large Doctor of Philosophy, 2019} \\
\ \\
Dissertation directed by: & {\large  Professor Peter J. Keleher} \\
&               {\large  Department of Computer Science } \\
\end{tabular}

\vspace{3em}

\renewcommand{\baselinestretch}{2}
\large \normalsize

The success of virtualization and container-based application deployment has fundamentally changed computing infrastructure from dedicated hardware provisioning to on-demand, shared clouds of computational resources. One of the most interesting effects of this shift is the opportunity to localize applications in multiple geographies and support mobile users around the globe. With relatively few steps, an application and its data systems can be deployed and scaled across continents and oceans, leveraging the existing data centers of much larger cloud providers.

The novelty and ease of a global computing context means that we are closer to the advent of an Oceanstore, an Internet-like revolution in personalized, persistent data that securely travels with its users. At a global scale, however, data systems suffer from physical limitations that significantly impact its consistency and performance. Even with modern telecommunications technology, the latency in communication from Brazil to Japan results in noticeable synchronization delays that violate user expectations. Moreover, the required scale of such systems means that failure is routine.

To address these issues, we explore consistency in the implementation of distributed logs, key/value databases and file systems that are replicated across wide areas. At the core of our system is hierarchical consensus, a geographically-distributed consensus algorithm that provides strong consistency, fault tolerance, durability, and adaptability to varying user access patterns. Using hierarchical consensus as a backbone, we further extend our system from data centers to edge regions using federated consistency, an adaptive consistency model that gives satellite replicas high availability at a stronger global consistency than existing weak consistency models.

In a deployment of 105 replicas in 15 geographic regions across 5 continents, we show that our implementation provides high throughput, strong consistency, and resiliency in the face of failure. From our experimental validation, we conclude that planetary-scale data storage systems can be implemented algorithmically without sacrificing consistency or performance.
