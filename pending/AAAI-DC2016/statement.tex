% Personal Statement
\documentclass[11pt,letterpaper]{article}

\usepackage[margin=1in]{geometry}
\usepackage[T1]{fontenc}
\usepackage[utf8]{inputenc}
\usepackage[parfill]{parskip}
\usepackage{titling}

% Title Fixes
\posttitle{\par\end{center}}
\setlength{\droptitle}{-65pt}

% Remove page numbers
\pagenumbering{gobble}

\title{Personal Statement}
\author{Benjamin Bengfort \textit{bengfort@cs.umd.edu}}
\date{\today}

\begin{document}

\maketitle

My graduate academic advisors have had extremely good fortune upon accepting me as their student; three of them were promoted to take larger roles at more prestigious universities, and one was offered a lucrative position at Bell Labs. Obviously many factors led to their success, but since 100\% of my advisors have had promotions within a year of advising me I suspect that I exist in some sort of tenure projection field that has aligned them well with their respective search committees. Or it could be that I simply have good taste in advisors. In either case, the departure of my advisors has led to a rather tumultuous academic career, observable in the wide breadth of topics of my academic publications.

It is perhaps for this reason that I have a hard time answering the question, ``what year are you?'' I am either in my third year or my eighth, depending on who you ask and the mood of the Graduate School. But what I can't deny, is that due to gaps, fits, and starts in my pursuit of the academic dream, I do have a more comprehensive work experience than some of my fellow, younger classmates. As a result, I'm much more comfortable answering the question ``why do you want your PhD?'' than they are, particularly as my graduate experience has not been to improve my opportunity to acquire a job (nor in the pursuit of a University position), but rather is an investment in myself so that I can pursue something that will change the world.

My professional career has been one of startups, large and small. Being in a startup is hard, it requires commitment, dedication, and a little bit of self-sacrfice. But the entrepreneurial spirit of taking risks to do something big, often makes those difficulties worth it. Although the fixtures are changed, pursuing a PhD requires that same entrepreneurial character, a willingness to be a jack-of-all-trades, to do what needs to be done if only to make a small contribution to human knowledge. I believe that the disruptive influence of startup culture, combined with my varied academic background has made me able to think and see new opportunities, integrations, and cross-disciplinary approaches that I have brought to my research at the University of Maryland.

Now at last I'm approaching the final stretch of my academic career. This semester I will have completed all of the course and residency requirements at the University of Maryland, where I have finally landed. I have a committee committed to my work, promotions aside, and I am preparing to propose a dissertation in the field of Artificial Intelligence at the end of next semester. Although a diverse background has brought me this far, it is time to focus towards an in-depth study where I can contribute my research; particularly a study of computational investigation and the trajectories of knowledge goals.

By participating in and presenting at the AAAI Doctoral Consortium, I hope to refine my study and research interests to a laser focus and to complement the advice of my academic advisor. I believe that the DC will be a critical step towards understanding my topic in the context of Artificial Intelligence research, and that the contextualization of my research will fine tune the path to the completion of my dissertation. I have not participated in any previous doctoral consortia, and I am not applying to any others in 2016 because I believe the AAAI DC is the correct place to present my research and the only one that will meet this goal. In return, I hope to bring my startup and entrepreneurial energy to the DC as an active and lively participant; encouraging my peers and engaging with more seasoned researchers. And perhaps we may even observe more effects of the tenure projection field!

%\bibliographystyle{plain}
%\bibliography{publications}

\end{document}
