\documentclass[sigconf]{acmart}

% ACM Packages
\usepackage{booktabs} % for formal tables

% Copyright
\setcopyright{rightsretained}
\copyrightyear{2017}

% DOI and ISBN
\acmDOI{http://dx.doi.org/10.1145/3087801.3087853}
\acmISBN{978-1-4503-4992-5/17/07}

% Conference
\acmYear{2017}
\acmConference{PODC '17}{July 25-27, 2017}{Washington, DC, USA}
\acmPrice{}

% Not sure what this does, but was told to include it in the instructions to authors.
\clubpenalty=10000
\widowpenalty = 10000

% Pete's editing special sauce
\usepackage{color}
\newcommand{\todo}[1]{{\textcolor{red}{#1}}}
\newcommand{\pjk}[1]{[\todo{PJK: #1}]}

\begin{document}

% Paper title must include prefix
\title{Brief Announcement: Hierarchical Consensus}

% acm author format
\author{Benjamin Bengfort}
\affiliation{%
  \institution{University of Maryland}
  \streetaddress{}
  \city{College Park}
  \state{Maryland}
  \postcode{20742}
}
\email{bengfort@cs.umd.edu}

\author{Pete Keleher}
\affiliation{%
  \institution{University of Maryland}
  \streetaddress{}
  \city{College Park}
  \state{Maryland}
  \postcode{20742}
}
\email{keleher@cs.umd.edu}
% end acm author format

% For a three page paper, we can probably omit the abstract.
% \begin{abstract}
% Groups of strongly consistent devices can efficiently replicate data under
ideal (data center) conditions, but become less effective in dynamic and
heterogeneous environments.  Eventually consistent devices efficiently
tolerate both faults and dynamic conditions but are slow to converge on a
single ordering of system events.

We propose ``federated consistency'', an approach that combines the strengths
of both approaches into a single protocol. Federated groups use a strongly
consistent inner core of devices to maintain a totally ordered, fault-tolerant
sequence of events.  A cloud of eventually-consistent devices disseminates
orderings and enables progress despite varying connectivity and partitions.
We use a detailed event simulation to show that a group of federated devices
can obtain the key advantages of both approaches. 

% \end{abstract}

% The code below should be generated by the tool at
% http://dl.acm.org/ccs.cfm
% NOTE: omit from paper, but include in upload to submission site. 
\begin{CCSXML}
<ccs2012>
<concept>
<concept_id>10002951.10003152.10003517.10003519</concept_id>
<concept_desc>Information systems~Distributed storage</concept_desc>
<concept_significance>500</concept_significance>
</concept>
<concept>
<concept_id>10002951.10003152.10003166.10003172</concept_id>
<concept_desc>Information systems~Remote replication</concept_desc>
<concept_significance>300</concept_significance>
</concept>
<concept>
<concept_id>10010520.10010521.10010537.10010539</concept_id>
<concept_desc>Computer systems organization~n-tier architectures</concept_desc>
<concept_significance>500</concept_significance>
</concept>
<concept>
<concept_id>10010520.10010575.10010577</concept_id>
<concept_desc>Computer systems organization~Reliability</concept_desc>
<concept_significance>500</concept_significance>
</concept>
<concept>
<concept_id>10010520.10010575.10010578</concept_id>
<concept_desc>Computer systems organization~Availability</concept_desc>
<concept_significance>300</concept_significance>
</concept>
<concept>
<concept_id>10010520.10010575.10011743</concept_id>
<concept_desc>Computer systems organization~Fault-tolerant network topologies</concept_desc>
<concept_significance>300</concept_significance>
</concept>
</ccs2012>
\end{CCSXML}

\ccsdesc[500]{Information systems~Distributed storage}
\ccsdesc[300]{Information systems~Remote replication}
\ccsdesc[500]{Computer systems organization~n-tier architectures}
\ccsdesc[500]{Computer systems organization~Reliability}
\ccsdesc[300]{Computer systems organization~Availability}
\ccsdesc[300]{Computer systems organization~Fault-tolerant network topologies}

% NOTE: omit from paper, but include in upload to submission site 
% NOTE: must be separated by semi-colons, not commas. 
%\keywords{consensus; consistency; leaders}

% NOTE: must also generate thumbnail image to upload to submission site

\maketitle

\section{Introduction}

% Strong consistency in a geo-replicated distributed data store requires a fault-tolerant
% mechanism that maintains consistency during node failure and communication partitions.
% Distributed consensus protocols inspired by Paxos \cite{lamport_paxos_2001} have been
% widely adopted to coordinate consistency, however, because of increased communication
% they cannot scale to arbitrary system sizes\cite{2016arXiv160806696H}.
% Several recent algorithms have attempted to address the scaling limitations of consensus
% and take two general forms --- leader election and conflict detection.
%
% Leader-oriented consensus such as MultiPaxos \cite{lamport_paxos_2001} and Raft
% \cite{ongaro_search_2014} minimize the number of required communication phases by
% nominating a dedicated proposer.
% Less communication means better throughput, and fault-tolerance is achieved through node
% failure detection such as a heartbeat and a new leader is elected to minimize downtime.
% Leader-oriented approaches introduce two new problems, however: \emph{load} and
% \emph{distance}.
% Because the leader will necessarily do more work and handle more communication than other
% nodes, it must have sufficient resources to handle the workload; moreover, since any node
% can be elected leader, all nodes must have sufficient resources to handle the workload.
% This introduces scaling problems in two dimensions: adding nodes means more communication,
% increasing the minimum resource requirements for all nodes in the system.
% In geo-replicated systems, bandwidth and latency are highly variable therefore the
% election of a leader in a specific location means that consensus is bound by the slowest
% connection, making the consensus algorithm sensitive to distance.
% Although recent approaches such as S-Paxos \cite{biely_s-paxos:_2012} and Mencius
% \cite{mao_mencius:_2008} add load balancing to leader-oriented mechanisms, they cannot
% solve the distance problem.
%
% Conflict detection approaches such as EPaxos \cite{moraru_there_2013} and MDCC
% \cite{kraska_mdcc:_2013} are optimistic that most consensus decisions are consistent.
% They propose ``fast'' and ``slow'' consensus paths, such that a subset of close nodes can
% quickly reach consensus but add dependency information to detect conflicts when commands
% are applied.
% If a conflict is detected, then decision must traverse the ``slow'' path to ensure
% correctness.
% Conflict detection does not have a distance problem, as nodes can select close neighbors,
% however this method does not guarantee dissemination of the command, which can require
% large amounts of dependency resolution.
% As the network scales, dependency graphs tend to increase in both size and complexity,
% increasing the load on the system.
%
% In practice systems do not implement global consensus, but instead apply multiple
% consensus groups to coordinate specific objects or tablets.
% This keeps quorum sizes small, allocating just enough nodes to a quorum to maintain a
% minimum level of fault tolerance.
% However, in so doing, an object can only be consistent with respect to its own updates
% and the system loses information about dependencies.
% Moreover, there is no coordination between consensus groups, a single node can
% participate in multiple per-object consensus groups, which does not eliminate node and
% distance problems.

% Probably need to hoist this paragraph up, and remove some of the problem statement
% stuff in the paras above.

We introduce \emph{Hierarchical Consensus}, an approach to generalizing
consensus that allows us to scale groups beyond a handful of nodes, across
wide areas.
Hierarchical Consensus (HC) increases the availability of consensus groups by
partitioning the decision space and nominating distinct leaders for each
partition.
Partitions eliminate distance by allowing decisions to be co-located with
replicas that are responding to accesses.
Hierarchical consensus is flexible locally, but improves upon prior
approaches~\cite{lamport_paxos_2001,2016arXiv160806696H,biely_s-paxos:_2012,mao_mencius:_2008,moraru_there_2013,kraska_mdcc:_2013}
by balancing load, allowing fast replication across wide areas, and enabling
consensus across large (\textgreater 100) systems of devices.

Our default use case is in maintaining \emph{linearizablity} across
\texttt{read} and \texttt{update} operations used to support a wide-area
object store or file system.
%
% \section{Consistency and Consensus}
%
We consider a set of processes $P = \{p_i\}_{i=1}^n$ which are connected via
an asynchronous network, whose connections are highly variable.
The variability of a communication link between $p_i$ and $p_j$ is modulated
by the physical distance of the link across the geographic wide area.
Each process maintains the state of a set of objects, $O = \{o_i^v\}_{i=1}^m$,
which are accessed singly or in groups at a given process and whose state is
represented by a monotonically increasing version number, $v$.
% There are two primary types of accesses, \texttt{read}, which returns the
% current versions of the accessed objects, and \texttt{update}, which
% increments the versions of the accessed objects.

% and wall clock time.
% \emph{Sequential consistency} allows for concurrent accesses so long as there
% is some globally defined ordering, which is specified by the
% \texttt{happens-before} relation ($\rightarrow$).
% Sequential consistency therefore only considers \texttt{update} operations but
% specifies an ordering of updates, maintaining a sequence $o_i^w \rightarrow
% o_i^{w+1} \rightarrow o_j^x$ and so forth.
% Objects that are accessed together (by the same process and within a defined
% window of time) are implicitly dependent on each other, requiring their
% relative access order to be strictly defined.
% Objects that are not implicitly dependent on each other, e.g.
% accessed by separate processes, do not require a strict ordering and can
% instead be arbitrarily ordered by process id.

% Distributed consensus protocols implement strong consistency by maintaining an
% ordered log of accesses (commands).
% Once a leader has been elected (a dedicated proposer), accesses are forwarded
% to the leader who, after checking application-specific invariants, broadcasts
% a request to other nodes to append the access to their log.
% A decision is reached when a majority of the quorum accept the append, at
% which point the leader sends a commit message.
% Fault-tolerance is observed by nodes that fall behind by replaying the log of
% committed accesses until they are up to date.
% Correctness is maintained by observing communications failure from the leader
% and electing a new leader.

\section{Hierarchical Consensus}

Hierarchical consensus conducts coordination decisions as a tier of quorums such that
parent quorums manage the \emph{decision space} and leaf quorums manage \emph{access ordering}.
Hierarchical consensus considers the decision space as subsets of the object set, and
subquorums are defined by a time-annotated disjoint subset of the objects they maintain,
$Q_{i,e} \subset O$.
The set of subquorums, $Q$, is not a complete partition of $O$, but only represents the set of
objects that are being accessed at time $e$.

% The E_i notation doesn't really match with the Q_{i,e} notation, not sure what to do.

The hierarchical consensus algorithm starts with a root quorum whose primary
responsibilities are i) the mapping of objects to subquorums and ii) the
mapping of replicas to subquorums.
Each instance of such a map defines a distinct \emph{epoch}, $e$, a
monotonically increasing representation of the term of $Q_{i,e}$.
Decisions that require a change of the decision space or changes the
mapping of objects or replicas to subquorums requires a new \emph{epoch}.
The fundamental relationship between epochs is as follows: any access that
happens in epoch $e \rightarrow e+1$ (happens before).
Alternatively, any access in epoch $e+1$ \emph{depends on} all accesses in
epoch $e$.

All accesses to an object must be forwarded to the leader of the subquorum that maintains
the object.
Objects that are accessed together or who have application-specific, explicit
dependencies (such as the set of objects included in a transaction) must be part of the
same subquorum so that local accesses are totally ordered.
Dependent objects that are not part of the same subquorum require either a change in
epoch or a mechanism to allow \emph{remote accesses}, which we will discuss in a
following section.
Accesses in different subquorums but in the same epoch happen concurrently
from the global perspective (but are non-conflicting), though accesses in a
specific subquorum are totally ordered locally.

\subsection{Operation}

The root consensus group coordinates all decision space changes.
Consider the simple example of the transfer of object responsibility from one subquorum
to another:

% Pete: I'm not a fan of this notation, was just following from what I had above. I think
% we could probably go back to the ABC --> ABGH notation.
\begin{quote}
\small
   $Q_1: \{o_a,o_b,o_c\} \rightarrowtail \{o_a,o_b,o_g,o_h\}$\\
   $Q_2: \{o_d,o_e,o_f,o_g,o_h\} \rightarrowtail \{o_c,o_d,o_e,o_f\}$
\end{quote}

Each of the two subquorums, $Q_1$ and $Q_2$, wants to give up a portion of its
existing decision space and to add objects currently mapped to another subquorum.
Reallocating subquorums requires a two phase consensus decision.
Both subquorum leaders send change requests to the leader of the parent quorum, which may
aggregate several requests into a single namespace change.
While the parent quorum gets consensus to make the epoch change, subquorums can continue
operating on their own decision space.
Once the parent quorum updates the epoch, it communicates the change to all affected
subquorums.
Each subquorum independently decides when to transition to the new epoch.
Subquorums in the new epoch can only access newly-gained objects once they have been
released by the objects' owners in the last epoch.

\subsection{Epochs and Ordering}

Hierarchical consensus requires all accesses in each subquorum to be linearizable,
guaranteed by serializing all accesses through the subquorum leader.
Global linearizability is guaranteed by serializing epochs at the parent
quorum, and limiting clients to access only one subquorum per epoch (relaxed
in Section~\ref{sec:remote}).

Let \emph{interval} $i_e$ be the ordered set of accesses of the replicas in subquorum
$Q_i$ during epoch $e$.
We enforce linearizable ordering of all accesses in the entire system by
ensuring that there must exist a total ordering of the intervals that produces the correct
access results.
Access results should be equivalent to any interval ordering
such that all intervals in $e$ occur before intervals in $e+1$ (our ``interval
ordering'' invariant).
This is because there is no cross-traffic between any $Q_i$ and $Q_j$, and therefore
ordering interval $i_e$ before $j_e$ is exactly the same as ordering interval $j_e$
before $i_e$, for any $i$, $j$, and $e$.

The internal invariant requires $\forall_{x,y} : Q_{x,e} \rightarrow Q_{y,e+1}$.
Ordering all accesses according to consensus-based log order and interval order satisfies both the
internal invariant and linearizability  while still allowing subquorums to operate
independently within epochs.
Given $Q_i$ and $Q_j$ within epochs $e=1$ and $e=2$, one possible interval order is
$Q_{i,1} \rightarrow Q_{j,1} \rightarrow Q_{i,2} \rightarrow Q_{j,2}$.

\subsubsection{Remote Accesses}
\label{sec:remote}

By default we assume that the set of replicas \emph{assigned} to subquorums are also
disjoint, and that all accesses through a replica of a given subquorum are mapped to the
local decision space.
This is often reasonable.
However, if an object is assigned to a decision space and a replica in another subquorum
wishes to
access it, the system must either disallow the access (our default approach) or take
explicit notice that a dependency has been created between the subquorums.

The latter approach requires a serialization of all accesses currently within the remote
quorum with respect to all accesses before the remote access in the local quorum.
Assume a read access from $Q_k$ to $Q_i$ in epoch $e$; at the time of the read all
accesses in $Q_i$ must $\rightarrow$ all accesses following the read.
We accommodate this requirement by using the read endpoints to break interval $i_e$ into
$i_{e.1}$ and $i_{e.2}$ at the point $Q_i$ receives the remote access, and interval $k_e$
into $k_{e.1}$ and $k_{e.2}$ at the point $Q_k$ receives a response as shown in
Figure \ref{fig:fuzzy}.
Our results are consistent with total interval ordering by incorporating $Q_{i.1}
\rightarrow Q_{i.2}$.
Remote accesses are expensive and the runtime system must weigh the cost of repeated
remote accesses against the cost of epoch changes.

\subsubsection{Fuzzy Epochs}

Only subquorums involved in decision space changes need take notice of
epoch changes.
Other subquorums can move to a new epoch at no cost when informed of new epoch
numbers from remote requests.
Slow-responding subquorums therefore do not block decision space changes for other
quorums.
% Because the subquorum is left behind in the previous epoch, all writes in that subquorum
% will be ordered before writes in the next epoch.
Safety is guaranteed because no writes in the next epoch depend on these writes.
These ``fuzzy epochs'' (Figure~\ref{fig:fuzzy}) allow an epoch change to be implemented solely by the
root quorum, allowing subquorums to move to the new epoch independently.
This flexibility is key to coping with partitions and varying connectivity in
the wide area.
% This means that subquorums can have ``fuzzy epochs'', wherein some subquorums are behind
% others.
% Fuzzy epochs provide the flexibility needed to accommodate subquorums that may not be
% ready to move to a new epoch eliminate because application semantics (still accessing the
% same objects), or network conditions (communication failure).
\begin{figure}[t]
    \centering
    \includegraphics[height=0.2\textheight]{figures/fuzzy}
    \caption{The gray region shows the ``fuzzy'' boundary between epochs $E_1$
      and $E_2$. The first read access, $r_k(x_{2.1})$, reads the value of object $x$ from $Q_i$ into
      $Q_k$ and forces $Q_k$ to change epochs.}
    \label{fig:fuzzy}
\end{figure}

% Figure~\ref{fig:fuzzy} shows three subquorums.
% The gray boundary delineates the border between epochs $E_1$ and $E_2$ -- it is fuzzy
% because not all subquorums move to epoch $E_2$ at the same time.
% The first read access, $r_k(x_{2.1})$, reads the value of object $x$ from $Q_i$ into
% $Q_k$.
% However, $Q_i$ is in $E_2$ when it services the read, while $Q_k$ is still in $E_1$ when
% the value is returned.
% If we follow the remote access approach, we can divide interval $i_2$ into $i_{2.1}$ and
% $i_{2.2}$, and $k_1$ into $k_{1.1}$ and $k_{1.2}$.
% We must also insure accesses are consistent with the interval ordering invariant, but
% this is not maintained because of the new dependency $i_{2.1} \rightarrow k_{1.1}$.
% Therefore, data coming from an interval in $E_n$ requires the receiver to also transition
% to $E_n$. This allows us to instead break our interval $i$ as before and maintain the
% ordering $i_{2.1} \rightarrow k_2$.

\section{Correctness Sketch}

We assert that consensus at the leaf nodes is correct and safe because decisions are
implemented using well-known leader-oriented consensus approaches.
% Because decision allocation occurs on accesses and is defined by a fixed period of time,
% we propose to show correctness through \emph{eventual quiescence}.
% Quiescence is the property that subquorums disband and return object ownership
% back to the parent quorum if activity ceases.
% Because all changes to the decision space require incrementing the epoch, if only the
% root quorum exists, the epoch is closed (e.g. no accesses will be applied to a log with
% that epoch).
Hierarchical consensus therefore has to demonstrate linearizable correctness and safety
between subquorums for a single epoch and between epochs.
Briefly, linearizability requires that external observers view operations to objects as
instantaneous events.
Within an epoch, the subquorum coordinates read and write accesses, and thus guarantees
linearizability for all replicas in that quorum.
Remote accesses and the internal invariant also enforce linearizability of accesses
between subquorums.
Epoch transitions raise the possibility of objects being re-assigned from one subquorum to
another, with each subquorum making the transition independently. Correctness is
guaranteed by an invariant that acquiring subquorums delay accessing newly acquired objects
until receiving notice that the releasing subquorum(s) have transitioned, \emph{plus} a
description of object versions at the point of this transition.
% Between epochs, the relationship of the parent quorum to subquorums defines correctness.
% The primary case to consider is an unsafe append to the access log: $Q_{i,e}$
% appends object $o_a^{v+1}$ while $Q_{j,e+1}$ appends object $o_a^v$ (incorrectly
% specifying that $o_a^{v+1} \rightarrow o_a^v$).
% It is the parent quorum's responsibility to ensure that $Q_{j,e+1}$ does not start
% operating until it has received confirmation from $Q_{i,e}$ that it has terminated.
% If the parent quorum does not receive a message from $Q_{i,e}$, it can force epoch $e$ to
% close at the latest known value, returning control to $Q_{j,e+1}$.
% This causes all accesses in $Q_{i,e}$ to be dropped when it communicates with the leader.
% Any quorum receiving messages from $Q_{i,e}$ will require $Q_i$ to move to epoch $e+1$,
% thus guaranteeing that no accesses will be appended after $Q_{j,e+1}$ has started
% operating.

\section{Experimental Design}

We propose to implement a distributed file system called FluidFS to more completely explore
the use of hierarchical consensus in supporting file systems.
FluidFS, implemented in the Go programming language, will allow us to quantitatively
describe real-world environments and usage and to show how our proposed consistency
model is experienced by users.

FluidFS will provide \emph{close-to-open consistency} (CTO), meaning that a file
open, which implies a full-file read, is guaranteed to see the data written by the ``most recent'' close of
that file.
Therefore file opens and closes must be totally ordered, and map
easily onto operations in a replicated log.
The canonical use of CTO is a single-server case, where a total
ordering of file open and closes is just the ordering that the opens
and closes arrive at the server.
The distributed case assumed by this work is much more demanding because
opens and closes are distributed across servers throughout the system.

We leverage hierarchical consistency to build a distributed set of sequentially-consistent logs
that guarantee a total ordering over all file accesses.
The result is a similar user experience to having the user/client co-located with a single
server, while the user migrates around the system, using different devices, possibly
collaborating with other users, and tolerating network vagaries, partitions, and failures.

Note that FluidFS, like many modern file systems,
decouples metadata file
data storage~\emph{recipes}
Metadata includes an ordered list of \emph{blobs}, which are opaque binary chunks.
When a file is closed after editing, the data associated with the file is chunked into a
series of variable-length blobs, identified by a hashing function applied to
the data.
The version created by the write access to the file specifies the blobs and their ordering
that make up the file.
Since blobs are effectively immutable, or tamper-evident, (blobs are named by hashes of
their contents), we assert that consistent metadata replication can be decoupled from blob
replication.
Accesses to file system metadata becomes the operations or entries in the replicated logs.
Metadata is therefore replicated through the system, allowing any file
system client to have a complete view of the file system namespace.

\section{Discussion}

Hierarchical consensus flexibly allocates subquorums to dynamic object groupings.
Multiple subquorums means both more leaders and less global communication, reducing the
resource requirements for most nodes, preventing bottlenecks, and increasing throughput.
Consensus decisions can also be localized to where the accesses are occurring,
minimizing both distance and the effect of wide area network variability.
Finally, hierarchical consensus does not arbitrarily assign consensus decisions to single
objects or unrelated groups of objects, but to objects that are implicitly dependent on
each other because of their associated accesses.

An open question for our research is how to automatically allocate the namespace such that
leadership of a subset of the namespace is local to the accesses and that members of the
quorum are distributed to provide wide area durability and availability.
To explore this question as well as empirically show the scalability of hierarchical
consensus, we are currently implementing a distributed file system called FluidFS.
FluidFS will allow us to quantitatively describe real world environments and usage and
to demonstrate how our proposed consistency model is experienced by users.

\bibliographystyle{ACM-Reference-Format}
\bibliography{paper}

\end{document}
