%Chapter 3

\renewcommand{\thechapter}{3}

\chapter{Hierarchical Consensus}
\label{ch:hierarchical_consensus}
% Overview, details, consistency, performance, lessons learned.

The backbone of our planetary scale data system is \emph{hierarchical consensus}~\cite{hc_brief_announcement}.
Hierarchical consensus provides a strong consistency foundation that totally orders critical accesses and arbitrates the eventual consistency layer in the fog, which raises the overall consistency of the system.
To be effective, an externalizable view of consistency ordering must be available to the entire system.
This means that strong consistency must be provided across geographic links rather than provided as localized, independent decision making with periodic synchronization.
The problem that hierarchical consensus is therefore designed to solve is that of geographically distributed consensus.

Solutions to geo-distributed consensus primarily focus on providing high throughput, low latency, fault tolerance, and durability.
Current approaches \cite{epaxos,mencius,calvindb,spaxos,sutra_fast_2011,peluso_making_2016} usually assume few replicas, each centrally located on a highly available, powerful, and reliable host.
These assumptions are justified by the environments in which they run: highly curated environments of individual data centers connected by dedicated networks.
Although replicas in these environments may participate in global consensus, our architecture requires us to accommodate replicas with heterogenous capabilities and usage modalities.
Widely distributed replicas might have neither high bandwidth nor low latency and might suffer partitions of varying durations.
Such systems of replicas might also be dynamic in membership, unbalanced across multiple regions to support variable sized workloads in each region.
Most importantly, to provide a backbone for a planetary scale data system, the consistency backbone must scale to include potentially thousands of replicas around the world.

As a result, straightforward approaches of running variants of Paxos~\cite{paxos,paxos_simple}, ePaxos~\cite{epaxos}, or Raft~\cite{raft} across the wide area, even for individual objects, will perform poorly for several reasons.
First, distance (in network connectivity) between the most active consensus replicas and their followers decrease the performance of the entire system; said another way, consensus is only as fast as the final vote required to make a decision, even when making ``thrifty'' requests.
Second, network partitions are common, which causes consensus algorithms to fail-stop if they cannot receive a majority~\cite{fail-stop}; a criticism that is often used to justify eventual consistency systems for high availability.
Finally, the fault tolerance model of small quorum algorithms can be disrupted by only a few unreliable hosts.
Given the scale of the system in question and the heterogenous nature of replicas, the likelihood of individual failure is so high so as to be considered inevitable, requiring much stronger fault tolerance guarantees.

We propose another approach to building large consensus systems.
Rather than relying on a few replicas to provide consensus to many clients, we propose to run a consensus protocol across replicas running at or near all of these locations.
The key insight is that large problem spaces can often be partitioned into mostly disjoint sets of activity without violating consistency.
We exploit this decomposition property by making our consensus protocol hierarchical so that the problem space encompasses the entire system and individual consensus groups fast by ensuring they are small and coordinate with other consensus groups only when required.
We exploit locality by building subquorums from colocated replicas, and locating subquorums near clients they serve, minimizing wide-area latency.
But because all replicas participate, the fault tolerance of a large system is realized.

In this chapter we describe hierarchical consensus, a tiered consensus structure that allows high throughput, localization, agility, and linearizable access to a shared namespace.
We show how to use \emph{delegation} (\S~\ref{ch03_delegation})  to build large consensus groups that retain their fault tolerance properties while performing like small groups.
We describe the use of \emph{fuzzy epoch transitions} (\S~\ref{ch03_fuzzy_transitions}) to allow global re-configurations across multiple consensus groups without forcing them into lockstep.
Finally, we describe how we reason about consistency by describing the structure of \emph{grid consistency} (\S~\ref{ch03_grid_consistency}).

\section{Overview}
\label{ch03_overview}

Hierarchical Consensus (HC) is a leader-oriented implementation and extension of Vertical Paxos~\cite{vertical_paxos,boxwood,niobe} designed to scale to hundreds of nodes geo-replicated around the world.
Vertical Paxos divides consensus decisions both horizontally, as sequences of consensus instances, and vertically as individual consensus decisions are made.
Spanner~\cite{spanner}, MDCC~\cite{mdcc}, and Calvin~\cite{calvindb}, can all be thought of as implementations of Vertical Paxos.
These systems shard the namespace of the objects they manage into individual consensus groups (the horizontal division) each independently reaching consensus about accesses (the vertical division).
A general implementation of Vertical Paxos therefore requires multiple independent quorums, one to manage the namespace and keep track of the health of all replicas in the system, and many other subquorums to independently manage tablets or objects.

In a geo-replicated context, there are two problems with this scheme.
First, sharding does not allow for inter-object dependence (in the horizontal division) without relying on coordination from the management quorum.
Second, the management quorum must be geo-replicated and able to scale to handle monitoring of the entire system.
In both cases, the management quorum becomes a bottleneck.
Current solutions to the bottleneck include batching decisions or using specialized hardware to produce extremely accurate timestamps.
These solutions, however, are outside of the scope of the safety properties provided by Vertical Paxos, which make it difficult to reason about consistency guarantees.
The challenge is therefore in constructing a multi-group coordination protocol that configures and mediates per-object quorums with the same level of consistency and fault tolerance as the entire system.

\begin{figure}
    \begin{center}
        \includegraphics[width=5in]{figures/ch03_hierarchical_topology.pdf}
    \end{center}
    \renewcommand{\baselinestretch}{1}
    \small\normalsize

    \begin{quote}
        \caption[A 12x3 Hierarchical Consensus Network Topology]{A simple example of an HC network composed of 12 replicas with size 3 subquorums. Each region hosts its own subquorum and subquorum leader, while the subquorum leaders delegate their votes to the root quorum, whose leader is found in region 1. This system also has 2 hot spares that can be used to quickly reconfigure subquorums that experience failures. The hot spares can either delegate their vote, or participate directly in the root quorum.}
        \label{fig:ch03_hierarchical_topology}
    \end{quote}
\end{figure}
\renewcommand{\baselinestretch}{2}
\small\normalsize

Hierarchical consensus organizes \emph{all} participating replicas into a single root quorum that manages the namespace across all regions as shown in Figure~\ref{fig:ch03_hierarchical_topology}.
The root quorum guarantees correctness and failure-free operation by pivoting the overall system through two primary functions.
First, the root quorum reconfigures its membership into subquorums, reserving extra members as hot-spares if needed.
Second, the root quorum adjusts the mapping of the object namespace defined by the application layer to the underlying partitions, which subquorums are responsible for managing.
These adjustments adapt the system to replica failures, system membership changes, varying access patterns, and ensure that related objects are coordinated together.
Much of the system's complexity comes from handshaking between the root quorum and the lower-level subquorums during reconfigurations.

% The root quorum elects a leader which partitions the namespace amongst its members into multiple subquorums, reserving extra members as hot-spares if needed.
% Subquorums elect their own leaders, which then coordinate accesses to the objects which they are assigned.

These handshakes are made easier, and much more efficient, by using \emph{fuzzy transitions}.
Fuzzy transitions allow individual subquorums to move through reconfiguration at their own pace, allowing portions of the system to transition to decisions made by the root quorum before others.
Given our heterogenous, wide-area environment, forcing the entire system to transition to new configurations in lockstep would be unacceptably slow.
Fuzzy transitions also ensure that there is no dedicated shard-master that has to synchronize all namespace allocations.
At the possible cost of multiple redirections, clients can be redirected by any member of the root quorum to replicas who should be participating in consensus decisions for the requested objects.
Fuzzy transitions ensure that root quorum decisions need not be timely since those decisions do not disrupt accesses of clients.

\begin{figure}
    \begin{center}
        \includegraphics[width=5in]{figures/ch03_hc_operation_summary.pdf}
    \end{center}
    \renewcommand{\baselinestretch}{1}
    \small\normalsize

    \begin{quote}
        \caption[HC Operational Summary]{A condensed summary of the hierarchical consensus protocol. Operations are described top-to-bottom where the top level is root quorum operations, the bottom is subquorum operations, and the middle is transition and intersection.}
        % Section numbers such as §5.2 indicate where particular features are discussed.
        \label{fig:ch03_hc_operation_summary}
    \end{quote}
\end{figure}
\renewcommand{\baselinestretch}{2}
\small\normalsize

Though root quorum decisions are rare with respect to the throughput of accesses inside the entire system, they still do require the participation of all members of the system, which could lead to extremely large quorum sizes, and therefore extremely slow consensus operations that may be overly sensitive to partitions.
Because all subquorums make disjoint decisions and because all members of the system are part of the root quorum, we propose a safe relaxation of the participation requirements for the root quorum such that subquorum followers can \emph{delegate} their root quorum votes to their leader.
Delegation ensures that only a few replicas participate in most root quorum decisions, though decisions are made for the entire system.

In brief, the resulting system is local, in that replicas serving clients can be located near them.
The system is fast because individual operations are served by small groups of replicas, regardless of the total size of the system.
The system is nimble in that it can dynamically reconfigure the number, membership, and responsibilities of subquorums in response to failures, phase changes in the driving applications, or mobility among the member replicas.
Finally, the system is consistent, supporting the strongest form of per-object consistency without relying on special-purpose hardware~\cite{fawn,corfu,vcorfu,tango,spanner}.

A complete summary of hierarchical consensus is described in Figure~\ref{fig:ch03_hc_operation_summary}.

\section{Consensus}
\label{ch03_consensus}

The canonical distributed consensus used by systems today is Paxos~\cite{paxos,paxos_simple}.
Paxos is provably safe and designed to make progress even when a portion of the system fails.
As described in \S~\ref{ch02_consistency}, consensus operations maintain a single, ordered log of operations that consistently change the state of the replica.
Raft~\cite{raft} was designed not to improve performance, but to increase understanding of consensus behavior to better allow efficient implementations.
HC uses Raft as a building block, so we describe the relevant portions of Raft at a high level, referring the reader to the original paper for complete details.
Though we chose to base our protocol on Raft, a similar approach could be used to modify Paxos or one of its variants into a hierarchical structure.

Consensus protocols have two phases: leader \emph{election} (also known as \texttt{PREPARE}) and operations \emph{commit} (also known as \texttt{ACCEPT}).
Raft is a strong-leader consensus protocol, a common optimization of Paxos variants called multi-Paxos~\cite{paxos_practical,paxos_live,paxos_systems_builders,spaxos}.
Multi-Paxos allows the election phase to be elided while a leader remains available, therefore the protocol requires only a single communication round to commit an operation in the common case.

Raft uses timeouts to trigger phase changes and provide fault tolerance.
Crucially, it relies on timeouts only to provide progress, not safety.
New elections occur when another replica in the quorum times out waiting for communication from the leader.
Such a replica increments its \emph{term} until it is greater than the existing leader, and announces its candidacy by sending a \texttt{VoteRequest}.
Other replicas vote for the candidate if they have not seen a competing candidate with a larger term and become followers, waiting for entries from the leader to be appended to its log.

During regular operation, clients send requests to the leader, which broadcasts \texttt{AppendEntries} messages carrying operations to all followers.
Term-invariants guarantee safety, followers will only accept an append entry request from a leader with a term as high or higher than the follower's term.
Additionally, followers will not append an entry to a log unless the leader can prove that the follower's log is as up to date as its own, determined by the index and term of the leader's previous entry.
An operation is \emph{committed} and can be executed when the leader receives acknowledgments of the \texttt{AppendEntries} message from more than half the replicas (including itself).

HC implements an adapted Raft consensus algorithm at both the root and subquorums.
We chose Raft both for its understandability in implementation and to easily describe operations in a leader-oriented context -- for example delegation is more understandable in the context of voting to elect a leader.
We describe differences in our Raft implementation from the canonical implementation in Chapter~\ref{ch:system_implementation}.

\subsection{Terminology and Assumptions}
\label{ch03_terminology}

Throughout the rest of this chapter we use the term \emph{root quorum} to refer to the upper, namespace-mapping and configuration-management tier of HC, and \emph{subquorum} to describe a group of replicas (called \emph{peers}) participating in consensus decisions for a section of the namespace.
The root quorum shepherds subquorums through \emph{epochs}, each with potentially different mappings of the namespace and replicas to subquorums.
An epoch corresponds to a single commit phase of the root quorum.
We use the term Raft only when describing details particular to our current use of Raft as the underlying consensus algorithm.
We refer to the two phases of the base consensus protocol as the \emph{election phase} and the \emph{commit phase}.
We use the term \emph{vote} as a general term to describe positive responses in either phase.

Epoch $x$ is denoted $e_x$ when referring to the numeric counter and $Q_x$ when referring to a configuration of subquorums.
Subquorum $i$ of epoch $e_x$ is represented as $q_{i,x}$, or just $q_i$ when the epoch is obvious, e.g. $q_i \in Q_x$.
The namespace is divided into \emph{tags}, disjoint subsets of the namespace.
It is equivalent to refer to $q_{i,x}$ as a tag since every subquorum manages at least one tag so that $Q_x$ covers the entire namespace.
However, for specificity we may refer to $t_{i,x}$ and $t_i \in T_x$ when referring to objects rather than the subquorums that manage them.
A system is composed of $R$ replicas, an individual replica may be designated $r_k$ to distinguish an individual replica from a subquorum.

We assume faults are fail-stop~\cite{fail-stop} rather than Byzantine~\cite{byzantine-generals}.
We do not assume that either replica hosts or networks are homogeneous, nor do we assume freedom from partitions and other network faults.


\subsection{Root Consensus}
\label{ch03_root_consensus}

Hierarchical consensus is a leader-oriented protocol that organizes replicas into two tiers of quorums, each responsible for fundamentally different decisions (Figure~\ref{fig:ch03_tiers}).
The lower tier consists of multiple independent \emph{subquorums}, each committing operations to local shared logs.
The upper, \emph{root quorum}, consists of subquorum peers, usually their leaders, delegated to represent the subquorum and hot spares in root elections and commits.
Hierarchical consensus's main function is to export a linearizable abstraction of shared accesses to some underlying substrate, such as a distributed object store or file system.
We assume that nodes hosting object stores, applications, and HC are frequently co-located across the wide area.

\begin{figure}
    \begin{center}
        \includegraphics[width=5in]{figures/ch03_election.pdf}
    \end{center}
    \renewcommand{\baselinestretch}{1}
    \small\normalsize

    \begin{quote}
        \caption[Root and Subquorum Composition]{The root quorum coordinates all replicas in the system including hot spares, though active participation is only by delegated representatives of subquorums, which do not necessarily have to be leaders of the subquorum, though this is most typical. Subquorums are configured by root quorum decisions which determine epochs of operation. Each subquorum handles accesses to its own independent portion of the namespace.}
        \label{fig:ch03_tiers}
    \end{quote}
\end{figure}
\renewcommand{\baselinestretch}{2}
\small\normalsize

The root quorum's primary responsibilities are mapping replicas to individual subquorums and mapping subquorums to tags within the namespace.
Each such map defines a distinct epoch, $e_x$, a monotonically increasing representation of the term of the configuration of subquorums and tags, $Q_x$.
The root quorum is effectively a consensus group consisting of subquorum leaders.
Somewhat like subquorums, the effective membership of the root quorum is not defined by the quorum itself, but in this case by leader election or peer delegations in the lower tier.
While the root quorum is composed of all replicas in the system, only this subset of replicas actively participates in root quorum decision making.

The root quorum partitions (shards) the namespace across multiple subquorums, each with a disjoint portion as its scope.
The namespace is decomposed into a set of tags, $T$ where each tag $t_i$ is a disjoint subset of the namespace.
Tags are mapped to subquorums in each epoch, $Q_x \mapsto T_x$ such that $\forall t \in T_x$ $\exists! q_{i,x} \mapsto t$.
The intent of subquorum localization is ensure that the \emph{domain} of a client, the portion of the namespace it accesses, is entirely within the scope of a local, or nearby, subquorum.
If true across the entire system, each client interacts with only one subquorum, and subquorums do not interact at all during execution of a single epoch.
This \emph{siloing} of client accesses simplifies implementation of strong consistency guarantees and allows better performance at the cost of restricting multi-object transactions.
We relax this restriction in \S~\ref{ch03_remote_accesses} to allow for the possibility of transactions.

\subsection{Delegation}
\label{ch03_delegation}

The root quorum's membership is, at least logically, the set of all system replicas, at all times.
However, running consensus elections across large systems is inefficient in the best of cases, and prohibitively slow in a geo-replicated environment.
Root quorum decision-making is kept tractable by having replicas \emph{delegate} their votes, usually to their leaders, for a finite duration of epochs.
With leader delegation, the root membership effectively consists of the set of subquorum leaders.
Each leader votes with a count describing its own and peer votes from its
subquorum and from hot spares that have delegated to it.

Fault tolerance scales with increasing system size and consensus leadership is intended to be stable.
A quorum leader is elected to indefinitely assign log entries to slots (access operations for subquorums, epoch configurations for the root quorum).
If the leader fails, then so long as the quorum has enough online peers, they can elect a new leader, and when the leader comes back online, it rejoins the quorum as a follower.
The larger the size of the quorum, the more failures it is able to tolerate.
This means that there might be multiple subquorum leaders in a single epoch as shown in Figure~\ref{fig:ch03_epochs_terms}.

Consider an alternative leader-based approach where root quorum membership is defined as the current set of subquorum leaders.
Both delegation and the leader approach have clear advantages in performance and flexibility over direct votes of the entire system.
However, the leader approach causes the root quorum to become unstable as its membership changes during partitions or subquorum elections.
These changes would require heavyweight \emph{joint consensus} decisions in the root quorum for correctness in Raft-like protocols~\cite{raft}.
By delegating at the root level, we introduce the possibility that a delegate fails, removing many root votes, to ensure root quorum stability (we address this possibility in \S~\ref{ch03_nuclear_option}).


\begin{landscape}
\begin{figure}
    \begin{center}
        \includegraphics[width=8.2in]{figures/ch03_epochs_terms.pdf}
    \end{center}
    \renewcommand{\baselinestretch}{1}
    \small\normalsize

    \begin{quote}
        \caption[Ordering of Epochs and Terms in Root and Subquorums]{The relationship of subquorum leadership (terms) to epochs is shown as follows. Three subquorums participate in HC in variable network conditions. In $e_1$, $q_{1,1}$ and $q_{3,1}$ experience outages causing them to elect new subquorum leaders while $q_{2,1}$ remains stable. After a root election transitions the system from $e_1 \rightarrow e_2$ (with a fuzzy transition from $q_{2,1} \rightarrow q_{2,2}$). The configuration $Q_2$ creates three totally new subquorums whose terms and leader election restarts.}
        \label{fig:ch03_epochs_terms}
    \end{quote}
\end{figure}
\renewcommand{\baselinestretch}{2}
\small\normalsize
\end{landscape}

Delegation ensures that root quorum membership is always the entire system and remains unchanged over subquorum leader elections and even reconfiguration.
Delegation is essentially a way to optimistically shortcut contacting every replica for each decision.
Subquorum repartitioning merely implies that a given replica's vote might need to be delegated to a different leader.
To ensure that delegation happens correctly and without requiring coordination, we simply allow a replica to directly designate another replica as its delegate until some future epoch is reached.
Replicas may only delegate their vote once per epoch and replicas are not required to delegate their vote.
To simplify this process, during configuration of subquorums by the root quorum, the root leader provides delegate hints, e.g. those replicas that have been stable members of the root quorum without partitions.
When replicas receive their configuration they can use these hints to delegate their vote to the closest nearby delegate if not already delegated for the epoch.
If no hints are provided, then replica followers generally delegate their vote to the term 1 leader and hot spares to the closest subquorum leader.

Delegation does add one complication: the root quorum leader must know all vote delegations to request votes when committing epoch changes.
We deal with this issue, as well as the requirement for a \emph{nuclear option} (\S~\ref{ch03_nuclear_option}), by simplifying our protocol.
Instead of sending vote requests just to subquorum leaders, \textbf{the root quorum leader sends vote requests to all system replicas.}
This is true even for \emph{hot spares}, which are not currently in any subquorum.
Delegates reply with the unique ids of the replicas they represent so that root consensus decisions are still made using a majority of all system replicas.

This is correct because vote requests now reach all replicas, and because replicas whose votes have been delegated merely ignore the request.
We argue that it is also efficient, as a commit's efficiency depends only on receipt of a majority of the votes.
Large consensus groups are generally slow (see Figure~\ref{fig:ch03_scaling_consensus}) not just because of communication latency, but because large groups in a heterogeneous setting are more likely to include replicas on very slow hosts or networks.
In the usual case for our protocol, the root leader still only needs to wait for votes from the subquorum leaders.
Leaders are generally those that respond more quickly to timeouts, so the
speed of root quorum operations is unchanged.

\subsection{Epoch Transitions}
\label{ch03_epoch_transitions}

% - tagspace (prefix-trie)
% - fuzzy transitions
% - anti-entropy and tombstones

Every epoch represents a new configuration of the system as designated by the root leader.
Efficient reconfiguration ensures that the system is both dynamic, responding both to failures and changing usage patterns, and minimizes coordination by colocating related objects.
An epoch change is initiated by the root leader in response to one of several events, including:

\renewcommand{\baselinestretch}{1}
\begin{itemize}
    \item a namespace repartition request from a subquorum leader
    \item notification of join requests by new replicas
    \item notification of failed replicas
    \item changing network conditions that suggest re-assignment of replicas
    \item manual reconfigurations, e.g. to localize data
\end{itemize}
\renewcommand{\baselinestretch}{2}

The root leader transitions to a new epoch through the normal commit phase in the root quorum.
The command proposed by the leader is an enumeration of the new subquorum partition, namespace partition, and assignment of namespace portions to specific subquorums.
The announcement may also include initial leaders for each subquorum, with the usual rules for leader election applying otherwise, or if the assigned leader is unresponsive.
Upon commit, the operation serves as an \emph{announcement} to subquorum leaders.
Subquorum leaders repeat the announcement locally, disseminating full knowledge of the new system configuration, and eventually transition to the new epoch by committing an \texttt{epoch-change} operation locally.

The epoch change is lightweight for subquorums that are not directly affected by the overarching reconfiguration.
If a subquorum is being changed or dissolved, however, the \emph{epoch-change} commitment becomes a tombstone written to the logs of all local replicas.4
No further operations will be committed by that version of the subgroup, and the local shared log is archived and then truncated.
Truncation is necessary to guarantee a consistent view of the log within a subquorum, as peers may have been part of different subquorums, and thus have different logs, during the last epoch.
Replicas then begin participating in their new subquorum instantiation.
In the common case where a subquorum's membership remains unchanged across the transition, an \texttt{epoch-change} may still require additional mechanism because of changes in namespace responsibility.

\subsection{Fuzzy Handshakes}
\label{ch03_fuzzy_transitions}

Epoch handshakes are required whenever the namespace-to-subquorum mapping changes across an epoch boundary.
HC separates epoch transition announcements in the root quorum from implementation in subquorums.
Epoch transitions are termed \emph{fuzzy} because subquorums need not all transition synchronously.
There are many reasons why a subquorum might be slow.
Communication delays and partitions might delay notification.
Temporary failures might block local commits.
A subquorum might also delay transitioning to allow a local burst of activity to cease such as currently running transactions\renewcommand{\baselinestretch}{1} \small\footnotesize\footnote{The HC implementation discussed in this chapter does not currently support transactions.}\renewcommand{\baselinestretch}{2} \small\normalsize.
Safety is guaranteed by tracking subquorum dependencies across the epoch boundary.

The most complex portion of the HC protocol is in handling data-related issues at epoch transitions.
Transitions may cause tags to be transferred from one subquorum to another, forcing the new leader to load state remotely to serve object requests.
Transition handshakes are augmented in three ways.
First, an replica can demand-fetch an object version from any other system replica.
Second, epoch handoffs contain enumerations of all current object versions, though not the data itself.
Knowing an object's current version gives the new handler of a tag the ability to demand fetch an object that is not yet present locally.
Finally, handshakes start immediate fetches of the in-core version cache from the leader of the tag's subquorum in the old epoch to the leader in the new.

\begin{figure}
    \begin{center}
        \includegraphics[width=5in]{figures/ch03_namespace_handoff.pdf}
    \end{center}
    \renewcommand{\baselinestretch}{1}
    \small\normalsize

    \begin{quote}
        \caption[Epoch Transition: Fuzzy Handshakes]{Readiness to transition to the new epoch is marked by a thin vertical bar; actual transition is the thick vertical bar.  Thick gray lines indicate operation in the previous epoch.  Subquorum $q_j$ transitions from tag ${c,d}$ to ${c,b,f}$, but begins only after receiving version information from previous owners of those tags.  The request to $q_k$ is only answered once $q_k$ is ready to transition as well.}
        \label{fig:ch03_namespace_handoff}
    \end{quote}
\end{figure}
\renewcommand{\baselinestretch}{2}
\small\normalsize

Figure~\ref{fig:ch03_namespace_handoff} shows an epoch transition where the scopes of $q_i$, $q_j$, and $q_k$ change across the transition as follows:

% TODO: Fix the alignment of the below equations

\renewcommand{\baselinestretch}{1}
\begin{eqnarray}
    q_{i,x-1} = t_a, t_b  &\longrightarrow q_{i,x} = t_a\\
    q_{j,x-1} = t_c, t_d  &\longrightarrow q_{j,x} = t_c,t_d,t_f\\
    q_{k,x-1} = t_e, t_f  &\longrightarrow q_{k,x} = t_d,t_e
\end{eqnarray}
\renewcommand{\baselinestretch}{2}

All three subquorums learn of the epoch change at the same time, but become ready with varying delays.
These delays could be because of network lags or ongoing local activity.
Subquorum $q_i$ gains no new tags across the transition and moves immediately to the new epoch.
Subquorum $q_j$'s readiness is slower, but then it sends requests to the owners of both the new tags it acquires in the new epoch.
Though $q_i$ responds immediately, $q_k$ delays its response until locally operations conclude.
Once both handshakes are received, $q_j$ moves into the new epoch, and $q_k$ later follows suit.

These bilateral handshakes allow an epoch change to be implemented incrementally, eliminating the need for lockstep synchronization across the entire system.
This flexibility is key to coping with partitions and varying connectivity in the wide area.
However, this piecewise transition, in combination with subquorum re-definition and configuration at epoch changes, also means that individual replicas \emph{may be part of multiple subquorums at a time}.

This overlap is possible because replicas may be mapped to distinct subgroups from one epoch to the next.
Consider $q_k$ in Figure~\ref{fig:ch03_namespace_handoff} again.
Assume the epochs shown are $e_x$ and $e_{x+1}$.
A single replica, $r_a$, may be remapped from subquorum $q_{k,x}$ to subquorum $q_{i,x+1}$ across the transition.
Subquorum $q_{k,x}$ is late to transition, but $q_{i,x+1}$ begins the new epoch almost immediately.
Requiring $r_a$ to participate in a single subquorum at a time would potentially delay $q_{i,x+1}$'s transition and impose artificial synchronicity constraints on the system.
One of the many changes we made in the base Raft protocol is to allow a replica to have multiple distinct shared logs.
Smaller changes concern the mapping of requests and responses to the appropriate consensus group.

\subsection{Subquorum and Client Operations}
\label{ch03_client_operations}

A subquorum, $q_{i,x}$, logically exists only for the duration of an epoch, $e_x$ and maps accesses to a subset of tags in $T$ such that $q_{i,x} \mapsto t_{i,x} \subset T_x$.
Each subquorum elects a leader to coordinate local decisions.
Fault tolerance of the subquorum is maintained in the usual way, detecting leader failures and electing new leaders from the peers.
Subquorums do not, however, ever change system membership on their own.
Subquorum membership is always defined in the root quorum.

Subquorum consensus is used to manage client accesses by committing object writes and designating a responding replica for object reads.
Clients can generally \texttt{Get} a key (a read operation), and can \texttt{Put} values and \texttt{Del} objects (write operations).
Client accesses are forwarded to the leader of the subquorum for the appropriate tag the object being accessed belongs to.
Because the root quorum manages the namespace, all replicas can correctly forward a client to a member of the subquorum, at worst requiring two redirects to reach a leader.
The underlying Raft semantics ensure that leadership changes do not result in loss of any commits.
Hence, individual- or multiple-client accesses to a single subquorum are totally ordered.

All writes are committed as operations by consensus decisions, appending the operation to a log shared by all replicas.
On commit, the write is applied to the underlying storage asynchronously, so long as the write is committed, it is guaranteed to be applied in the order specified by the log.
The shared logs also provide a complete version history of all distributed objects.
Subquorum leaders use in-core caches to provide fast access to recently accessed objects in the local subquorums's tag.
Replicas perform background anti-entropy~\cite{dynamo,bayou,anti_entropy}, disseminating log updates a user-defined number of times across the system, providing both durability as well as fast transitions between configurations.

Reads are not committed with consensus decisions by default.
Leaders respond to \texttt{Get} requests by replying with the latest applied (committed) value.
This introduces the possibility of a stale read, e.g. that a read occurs before a committed operation is applied.
To ensure a subquorum has linearizable consistency, reads would also need to be committed.
Another option is to commit read-leases to a specified client, so that they are guaranteed to read a snapshot of object values.
Committing multi-object leases is the basis for implementing transactions in subquorums, however they also play an important role in accessing multiple objects across subquorum boundaries.

Although we have not yet implemented transactions in our system, we have provided a mechanism for multi-object coordination in the case where objects span multiple subquorums.
A \emph{remote access} is conducted from one subquorum to another (e.g. leader to leader) transparently from the client.
Remote accesses have implications for consistency as described in \S~\ref{ch03_grid_consistency}.
For now we simply point out that all remote accesses, both reads and writes, require the subquorum to commit the operation or grant a temporary lease to the remote quorum.
To optimize this process, batch commands may be used to limit the amount of cross-region communication required for remote accesses.

\section{Consistency}
\label{ch03_consistency}

Hierarchical consensus provides the strongest possible per-object and global consistency guarantees.
Pushing all writes through subquorum commits and serving reads at leaders allows us to guarantee that per-object accesses are linearizable (Lin), which is the strongest non-transactional consistency~\cite{linearizability,sequential_consistency}.
As a recap, linearizability is a combination of atomicity and timeliness guarantees about accesses to a single object.
Both \texttt{reads} and \texttt{writes} must appear atomic, and also instantaneous at some time between a request and the corresponding response to a client.
\texttt{Reads} must always return the latest value.
This implies that reads return values are consistent \emph{with any observed ordering}, i.e., the ordering is \emph{externalizable}~\cite{externalizable}.

Linearizability of object accesses can be \emph{composed}.
If operations on each object are linearizable, the entire object space is also linearizable.
This allows our subquorums to operate independently while providing a globally consistent abstraction.
The resulting consistency model can be reasoned about as \emph{grid consistency}

\subsection{Grid Consistency}
\label{ch03_grid_consistency}

Hierarchical consensus implements a multi-quorum model of consistency we call grid consistency.
Grid consistency uses the Vertical Paxos three-dimensional log model to also model a global ordering of operations that ensure the state of the system is always consistent.
We express total orderings as ``happened-before'' ($\rightarrow$) relationships~\cite{lamport_time_1978}.
The grid is defined along the horizontal by each epoch or configuration.
Epochs are totally ordered, such that $e_{x-1} \rightarrow e_{x}$, determined by the root log's consensus operations.
This implies that any access in $q_{i,x-1} \rightarrow q_{j,x}$, which is guaranteed by the tombstone and hand-off operation during epoch transition.
The grid is defined vertically by logs of all $q_i \in Q_x$.
Because subquorums operate independently with the exception of remote accesses, operations in each log can be applied concurrently.
Said another way, all subquorum logs within an epoch implement a fuzzy log~\cite{fuzzy_log}.
In this section we show how remote accesses and independent subquorums in a single epoch create a linearizable total ordering via composability and in the following section we describe how to extract a sequentially consistent global log.

\begin{figure}
    \begin{center}
        \includegraphics[width=5in]{figures/ch03_event_ordering.pdf}
    \end{center}
    \renewcommand{\baselinestretch}{1}
    \small\normalsize

    \begin{quote}
        \caption[Sequential Event Ordering in HC]{Without remote accesses, once an epoch has been concluded a default total ordering is: $w_{i,1}\rightarrow w_{i,3}\rightarrow w_{j,1}\rightarrow w_{j,3}$. Once the epoch is concluded, this ordering is guaranteed and exposes an externalizable total ordering of events.}
        \label{fig:ch03_event_ordering}
    \end{quote}
\end{figure}
\renewcommand{\baselinestretch}{2}
\small\normalsize

Figure~\ref{fig:ch03_event_ordering} shows a system with subquorums $q_i$ and $q_j$, each of which performs a pair of writes.
Dotted lines show one possible event ordering for replicas $q_i$ (responsible for objects $a$ and $b$), and $q_j$ ($c$ and $d$).
Without cross-subquorum reads or writes, ordering either subquorums's operations first creates a SC total ordering: $q_i \rightarrow q_j$ implies $w_{i,1} \rightarrow w_{i,3} \rightarrow w_{j,1} \rightarrow w_{j,3}$, for example.
If the epoch has not been concluded, then the best the system can guarantee is sequential consistency, we can only guarantee that $w_{j,1} \rightarrow w_{j,3}$ and that $w_{i,1} \rightarrow w_{i,3}$.
Once the epoch is concluded, however, there is no other ordering other than the one expressed by the grid -- therefore at the epoch's conclusion a linearizable total ordering exists, though it cannot be read from.
However, because only a single subquorum will handle reads within an epoch (no other quorum will issue a read unless the epoch has changed) we posit that a client that reads only objects handled by a single subquorum has a linearizable view of those objects.
Because linearizability is composable across objects, the system is linearizable.

\subsection{Remote Accesses}
\label{ch03_remote_accesses}

The constraint that clients access only a single subquorum per epoch is acceptable in the common case, but because epoch changes require root quorum decisions, we allow cross-quorum communication via \emph{remote accesses}.
Figure~\ref{fig:ch03_event_ordering_remote_write} shows additional dependencies created by issuing remote writes to other subquorums: $w_{i,2} \rightarrow w_{j,3}$ and $w_{j,2} \rightarrow w_{i,3}$.
Each remote write establishes a partial ordering between events of the sender before the sending of the write, and writes by the receiver after the write is received.
Similar dependencies result from remote reads.

\begin{figure}
    \begin{center}
        \includegraphics[width=5in]{figures/ch03_event_ordering_remote_write.pdf}
    \end{center}
    \renewcommand{\baselinestretch}{1}
    \small\normalsize

    \begin{quote}
        \caption[Event Ordering with Remote Writes in HC]{Remote writes add additional ordering constraints: $w_{i,1} \rightarrow w_{i,2}\rightarrow w_{j,3}$, and $ w_{j,1} \rightarrow w_{j,2} \rightarrow w_{i,3}$. By creating \emph{subepochs}, we can guarantee linearizability even for accesses across multiple subquorums.}
        \label{fig:ch03_event_ordering_remote_write}
    \end{quote}
\end{figure}
\renewcommand{\baselinestretch}{2}
\small\normalsize

These dependencies cause the epochs to be logically split (not shown in picture).
The receipt of write $w_{i,2}$ in $q_j$ causes $q_{j,1}$ to be split into $q_{j,1.1}$ and $q_{j,1.2}$.
Likewise, the receipt of write $w_{j,2}$ into $q_i$ causes $q_i$ to be split into $q_{i,1.1}$ and $q_{i,1.2}$.
Any topological sort of the subepochs that respects these orderings, such as $q_{i,1.1} \rightarrow q_{j,1.1} \rightarrow q_{j,1.2} \rightarrow q_{i,1.2}$, results in a valid SC ordering.

As before, presenting a sequentially consistent global log across the entire system, then, only requires tracking these inter-subquorum data accesses, and then performing an $\mathcal{O}(n)$ merge of the subepochs.
By definition, this log's ordering respects any externally visible ordering of cross-subquorum accesses (accesses visible to the system).
So long as these dependencies are maintained, then the log is externalized at epoch boundaries.

However, the log does not necessarily order other accesses according to external visibility.
Extracting a global log could not be mined to find causal relationships between accesses through external communication paths unknown to the system.
For example, assume that log events are published posts, and that one user claimed plagiarism.
The accused would not be able to prove that his post came first unless there were some causal chain of posts and references visible to the protocol.

\subsection{Globally Consistent Logs}
\label{sec:ch03_log_ordering}

Our default use case is in providing linearizable access to an object store.
Though this approach allows us to guarantee all observers will see linearizable results of object accesses in real-time, the system is not able to enumerate a total order or create a linearizable shared log.
Such a linear order would require fine-grained (expensive) coordination across the entire system or fine-grained clock synchronization~\cite{spanner}.
Though many or most distributed applications (objects stores, file systems, etc.) will work directly with HC, shared logs are a useful building block for distributed systems.

HC \emph{can} be used to build a sequentially consistent (SC) shared log as shown in Figure~\ref{fig:ch03_log_ordering}.
Because SC does not require an externalizable total ordering, it merely has to conform to local operation orders and all reads-from dependencies created by remote accesses.
Therefore to construct a global log in real time, clients simply have to read from the logs of the subquorums, appending new entries as needed.
If a client gets to a remote access in the log of a subquorum, it must read from the remote subquorums log until the access has completed, before continuing to append entries of the current log.
This operation can be parallelized across multiple subquorum logs, requiring synchronization only at remote access points.

\begin{landscape}
\begin{figure}
    \begin{center}
        \includegraphics[width=8.2in]{figures/ch03_log_ordering.pdf}
    \end{center}
    \renewcommand{\baselinestretch}{1}
    \small\normalsize

    \begin{quote}
        \caption[Grid Consistency: A Sequential Log Ordering]{A sequential consistency view of a a global grid ordering. This figure shows the grid consistency model and happens before relationships between epoch configurations and writes between logs. This grid only exposes sequential consistency across all objects, however, because log operations are concurrent across all rows.}
        \label{fig:ch03_log_ordering}
    \end{quote}
\end{figure}
\renewcommand{\baselinestretch}{2}
\small\normalsize
\end{landscape}

We do not currently gather the entire shared log onto a single replica because of capacity and flexibility issues.
Capacity is limited because our system and applications are expected to be long-lived.
Flexibility is a problem because HC, and applications built on HC, gain much of their value from the ability to pivot quickly, whether to deal with changes in the environment or for changing application access patterns.
We require handoffs to be as lightweight as possible to preserve this advantage.

Instead, we propose a checkpoint strategy based on epochs.
Global logs may be used to recover side by side systems (such as a development environment) or for auditing purposes.
Once an epoch has been fully concluded the state of the system remains constant and can be represented by a version-vector of metadata for each object.
Auditing or recovery operations can then apply ongoing log accesses to the checkpoint state either for the entire system or simply a portion of the system.
Once the desired state has been reached, the version-vector can then be used to extract the specific objects or history required.
We utilized this method during development to provide consistency verification auditing across the system.

\section{Safety and Correctness}
\label{ch03_safety}

Distributed consensus requires provable safety and correctness so as to be relied upon when building consistency-centric systems.
Safety ensures that any update to the system, if committed, will be represented by the system even in the case of limited failure~\cite{safety}.
Correctness is described by the consistency model, if a distributed algorithm always produces an expected system state, then it is correct~\cite{correctness}.
Safety and correctness are proved as part of the scientific process of introducing new consensus algorithms.
Although we view hierarchical consensus as a consensus protocol rather than a new consensus algorithm, we provide a brief overview of our safety proof as follows.

We assert that consensus at individual subquorums is correct and safe because decisions are implemented using well-known leader-oriented consensus approaches.
Hierarchical consensus therefore has to demonstrate linearizable correctness and safety between subquorums for a single epoch and between epochs.
Briefly, linearizability requires external observers to view operations to objects as instantaneous events.
Within an epoch, subquorum leaders serially order local accesses, thereby guaranteeing linearizability for all replicas in that quorum.
Remote accesses and the internal subepoch invariant also enforce linearizability of accesses between subquorums inside of a single epoch as described in \S~\ref{ch03_grid_consistency}.
Given a static system of subquorum configurations that each manage independent shards, we claim that our system implements vertical paxos.

A static configuration would not require a root quorum, but it would also not allow reconfiguration to move quorums to locales of access or to repair system failures.
Therefore to prove safety and correctness, we must show that root quorum behavior, specifically epoch transitions and delegation, is correct.
Epoch transitions raise the possibility of portions of the namespace being re-assigned from one subquorum to another, with each subquorum making the transition independently.
Correctness is guaranteed by an invariant requiring subquorums to delay serving newly acquired portions of the namespace until after completing all appropriate handshakes.
Tombstones ensure that an update cannot be applied to a subquorum then lost when the transitioning subquorum takes over.
Delegation is protected by bookkeeping that ensures that no replica can be counted twice in a vote, therefore in the worst case, delegation means that a single failure can eliminate many votes.
% We propose formal specifications in Appendix~\ref{apx:formal_specification} to prove that these invariants are sufficient for correctness.

Safety and correctness are important parts of distributed consensus, but only if they also allow a system to make progress in the event of failure.
We define the system's \emph{safety} property as guaranteeing that non-linearizable (or non-sequentially-consistent, see Section~\ref{sec:ch03_log_ordering}) event orderings can never be observed.
We define the system's \emph{progress} property as the system having enough live replicas to commit votes or operations in the root quorum.
In the rest of this section, we will specifically identify types of expected failures that may harm our proposed guarantees and what amount of failure is tolerated before preventing progress.
We then describe two additional mechanisms that we use to ensure the safety of hierarchical consensus: the \emph{nuclear option} and \emph{obligation leases}.

\subsection{Fault Tolerance}
\label{ch03_fault_tolerance}

The system can suffer several types of failures, as shown in Table~\ref{tab:failure_categories}.
Both subquorum leaders and the root leaders send periodic beacons and heartbeat messages to their followers.
If a heartbeat message is missed, e.g. if a follower does not receive an expected heartbeat from its leader or if a leader does not receive a response from the heartbeat, then the system takes action to ensure it's still available to respond to clients.
Failures of subquorum and root quorum leaders are handled through the normal consensus mechanisms and a new leader is elected.
Failures of subquorum peers are handled by the local leader petitioning the root quorum to re-configure the subquorum in the next epoch.
Failure of a root quorum peer is the failure of a subquorum leader with delegated votes, this can be handled by a reconfiguration which reallocates the delegated votes to a new peer so long as a majority of delegates are available in the root quorum.
Root quorum beacon messages help inform replicas of leadership and configuration changes, which ensures the system adapts to temporary outages and failures.

\renewcommand{\baselinestretch}{1}
\small\normalsize
 \begin{table}[t!]
\caption[HC Failure Categories]{Failures include either node failure or network partitions which are detected by missed beacon or heartbeat messages. A replicas role and a threshold for the number of missed messages determines how the system responds.}
\begin{center}
\begin{tabular}{l|l}
\hline
Failure Type & Response \\
\hline \hline
subquorum peer & request reconfiguration from root quorum \\
subquorum leader & local election, request replacement from root quorum \\
subquorum & reconfiguration after obligations timeout \\
root leader & root election (with delegations)\\
majority of delegates & delegations time out (nuclear option) \\
\hline
\end{tabular}
\end{center}
\label{tab:failure_categories}
\end{table}
 \renewcommand{\baselinestretch}{2}
\small\normalsize

HC's structure means that some faults are more important than others.
Proper operation of the root quorum requires the majority of replicas in the majority of subquorums to be non-faulty.
Given a system with $2m+1$ subquorums, each of $2n+1$ replicas, the entire system's progress can be halted with as few as $(m+1)(n+1)$ well-chosen failures -- e.g. the assassination of the delegates.
Therefore, in worst case, the system can only tolerate: $f_{worst}=mn+m+n$ failures and still make progress.
At maximum, HC's basic protocol can tolerate up to: $f_{best} = (m+1)*n + m*(2n+1) = 3mn+m+n$ failures.
As an example a 25/5 system, that is a system of 25 replicas with size 5 subquorums ($m=2, n=2$), can tolerate at least 8 and up to 16 failures.
A 21/3 system can tolerate at least 7, and a maximum of 12, failures out of 21 total replicas.

To tolerate the most faults, the root quorum operates strategically to handle failures.
For example, individual subquorums might still be able to perform local operations despite an impasse at the global level.
The root quorum chooses carefully whether a failure type should involve a reconfiguration or whether the system should wait for an outage to be repaired.

There are two primary types of failures though that have to be dealt with specifically.
Total subquorum failure can temporarily cause a portion of the namespace to be unserved (or only served locally).
In this case we use obligation timeouts to determine when the root quorum should force a configuration change.
Additionally in the face of delegate assassination, where no root quorum decisions can be made, we use the nuclear option to eliminate delegates and require every replica to contribute their own votes.

\subsection{The Nuclear Option}
\label{ch03_nuclear_option}

Singleton consensus protocols, including Raft, can tolerate just under half of the entire system failing.
As described above, HC's structure makes it more vulnerable to clustered failures.
Therefore we define a \emph{nuclear option}, which uses direct consensus among all system replicas to tolerate any $f$ replicas failing out of $2f+1$ total replicas in the system.

A nuclear vote is triggered by the failure of a root leader election.
A \emph{nuclear candidate} increments its term for the root quorum and broadcasts a request for votes to all system replicas.
The key difficulty is in preventing delegated votes and nuclear votes from reaching conflicting decisions.
Such situations might occur when temporarily unavailable subquorum leaders regain connectivity and allow a wedged root quorum to unblock.
Meanwhile, a nuclear vote might be concurrently underway.

Replica delegations are defined as intervals over specific slots.
Using local subquorum slots would fall prey to the above problem, so we define delegations as a small number (often one) of root slots, which usually correspond to distinct epochs.
During failure-free operation, peers delegate to their leaders and are all represented in the next root election or commit.
Peers then renew their delegations to their leaders by appending them to the next local commit reply.
This approach works for replicas that change subquorums over an epoch boundary, and even allows peers to delegate their votes to arbitrary other peers in the system (see replicas $r_N$ and $r_O$ in Figure~\ref{fig:ch03_tiers}).

This approach is simple and correct, but deals poorly with leader turnovers in the subquorum.
Consider a subquorum where all peers have delegated votes to their leader for the next root slot.
If that leader fails, none of the peers will be represented.
We finesse this issue by re-defining such delegations to count root elections, root commits, \emph{and} root heartbeats.
The latter means that local peers will regain their votes for the next root quorum action if it happens after the next heartbeat.

Consider the worst-case failure situation discussed in \S~\ref{ch03_fault_tolerance}: a majority of the majority of subquorums have failed.
None of the failed subquorum leaders can be replaced, as none of those subquorums have enough local peers.

The first response is initiated when a replica holding delegations (or its own vote) times out waiting for the root heartbeat.
That replica increments its own root term, adopts the prior system configuration as its own, and becomes a root candidate.
This candidacy fails, as a majority of subquorum leaders, with all of their delegated votes, are gone.
Progress in the root quorum is not made until delegations time out.
In our default case where a delegation is for a single root event, this happens after the first root election failure.

At the next timeout, any replica might become a candidate because delegations have lapsed (under our default assumptions above).
Such a nuclear candidate increments its root term and sends candidate requests to all system replicas, succeeding if it gathers a majority across all live replicas.

The first candidacy assumed the prior system configuration in its candidacy announcement.
This configuration is no longer appropriate unless some of the ``failed'' replicas quickly regain connectivity.
Before the replica announces its candidacy for a second time, however, many of the replica replies have timed out.
The candidate alters its second proposed configuration by recasting all such replicas as hot spares and potentially reducing the number and size of the subgroups.
Subsequent epoch changes might re-integrate the new hot spares if the replicas regain connectivity.

\subsection{Obligations Timeout}
\label{ch03_obligations_timeout}

The second type of specific failure the root quorum must deal with is a subquorum stranded behind a network partition.
In this case the subquorum may be operating and serving local requests but its leader (or delegate) is unable to communicate to the root quorum.
In the majority of cases, the root quorum should wait out the presumably temporary system partition if client requests are being served.
However, it is also possible that all replicas in the subquorum have failed due to cascading correlated failure and no accesses to that portion of the namespace are being granted.

We therefore propose a configurable obligations timeout.
Subquorums are considered obligated to serve requests from clients for the duration of the epoch in which the subquorum is configured.
However to ensure that subquorums are in fact meeting those obligations, we introduce another timeout during which the subquorum has to communicate with the root leader, reconfirming its obligation for the next period.
If the obligation period times out without being renewed, the subquorum is obligated to stop handling client requests and the root quorum is obligated to reconfigure the system.

The problem is that the subquorum that is reallocated that portion of the namespace presumably would not be able to achieve a hand-off with the partitioned system.
It is also possible that the region the subquorum was configured in simply cannot be reconfigured through a root consensus decision.
In this case, there would be an unacceptable period of unavailability for that portion of the namespace.

To deal with this situation, both the newly configured subquorum and the previous subquorum must go into an ``unstable'' state, informing clients that their writes are not guaranteed the level of consistency the system normally provides.
Using a federated consistency model (discussed in Chapter~\ref{ch:federated_consistency}), replicas would simply assume a lower level of consistency.
An unstable mode of operation is repaired in one of two cases.
First, the partitioned subquorum may come back on line and is able to automatically negotiate the epoch transition, fixing conflicts where necessary.
Otherwise, the root quorum must manually determine that the subquorum had been destroyed, which results in data loss anyway.
Optimizations such as anti-entropy replication (described in \S~\ref{ch04_anti_entropy}) of data across regions and global views of data versions minimize the impact of such loss, but cannot prevent it.

\section{Performance Evaluation}
\label{ch03_evaluation}

HC was designed to adapt both to dynamic workloads as well as variable network conditions.
We therefore evaluate HC in three distinct environments: a homogeneous data center, a heterogeneous real-world network, and a globally distributed cloud network.
The homogeneous cluster is hosted on Amazon EC2 and includes 26 ``t2.medium'' instances: dual-core virtual machines running in a single VPC with inter-machine latencies ($\lambda$) normally distributed with a mean, $\lambda_{\mu}=0.399ms$ and standard deviation, $\lambda_{\sigma}=0.216ms$.
These machines are cost-effective and, though lightweight, are easy to scale to large cluster sizes as workload increases.
Experiments are set up such that each instance runs a single replica process and multiple client processes.

The heterogeneous cluster (UMD) consists of several local machines distributed across a wide area, with inter-machine latencies ranging from
$\lambda_{\mu}=2.527ms$,
$\lambda_{\sigma}=1.147ms$ to $\lambda_{\mu}=34.651ms$,
$\lambda_{\sigma}=37.915ms$.
Machines in this network are a variety of dual and quad core desktop servers that are solely dedicated to running these benchmarks.
Experiments on these machines are set up so that each instance runs multiple replica and client processes co-located on the same host.
In this environment, localization is critical both for performance but also to ensure that the protocol can elect and maintain consensus leadership.
The variability of this network also poses challenges that HC is uniquely suited to handle via root quorum-guided adaptation.
We explore two distinct scenarios -- sawtooth and repartitioning -- using this cluster; all other experiments were run on the EC2 cluster.

In our final experiment, we explore the use of hierarchical consensus in an extremely large, planetary-scale system comprised of 105 replicas in 15 data centers in 5 continents spanning the northern hemisphere and South America. This experiment was also hosted on EC2 ``t2.medium`` instances in each of the regions available to us at the time of this writing. In this context, reporting average latencies is difficult as inter-region latencies depend more on network distance than can be meaningfully ascribed to a single central tendency.

\begin{figure}
    \begin{center}
        \includegraphics[width=5in]{figures/ch03_scaling_consensus.pdf}
    \end{center}
    \renewcommand{\baselinestretch}{1}
    \small\normalsize

    \begin{quote}
        \caption[Scaling Consensus HC vs. Raft]{Mean throughput of workloads of up to 120 concurrent clients}
        \label{fig:ch03_scaling_consensus}
    \end{quote}
\end{figure}
\renewcommand{\baselinestretch}{2}
\small\normalsize

HC is partially motivated by the need to scale strong consistency to large cluster sizes.
We based our work on the assumption that consensus performance decreases as the quorum size increases, which we confirm empirically in Figure~\ref{fig:ch03_scaling_consensus}.
This figure shows the maximum throughput against system size for a variety of workloads, up to 120 concurrent clients.
A workload consists of one or more clients continuously sending writes of a specific object or objects to the cluster without pause.

Standard consensus algorithms, Raft in particular, scale poorly with uniformly decreasing throughput as nodes are added to the cluster.
Commit latency increases with quorum size as the system has to wait for more responses from peers, thereby decreasing overall throughput.
Figures~\ref{fig:ch03_scaling_consensus} and~\ref{fig:ch03_hc_throughput_workload} clearly show the multiplicative advantage of HC's hierarchical structure.
Note that though HC is not shown to scale linearly in these figures, this is due to performance bottlenecks of the networking implementation in these experiments.
In our final experiment, we show linear scaling with our latest implementation of HC.

\begin{figure}
    \begin{center}
        \includegraphics[width=5in]{figures/ch03_hc_throughput_workload.pdf}
    \end{center}
    \renewcommand{\baselinestretch}{1}
    \small\normalsize

    \begin{quote}
        \caption[HC Throughput vs. Workload in the Wide Area]{Performance of distributed consensus with an increasing workload of concurrent clients. Performance is measured by throughput, the number of writes committed per second.}
        \label{fig:ch03_hc_throughput_workload}
    \end{quote}
\end{figure}
\renewcommand{\baselinestretch}{2}
\small\normalsize

There are at least two factors limiting the HC throughput shown in our initial experiments.
First, the HC subquorums for the larger system sizes are not saturated.
A single 3-node subquorum saturates at around 25 clients and this experiment has only about 15 clients per subquorum for the largest cluster size.
We ran experiments with 600 clients, saturating all subquorums even in the 24-node case.
This throughput peaked at slightly over 50,000 committed writes per second, better but still lower than the linear scaling we had expected.

We think the reason for this ceiling is hinted at by Figure~\ref{fig:ch03_hc_throughput_workload}.
This figure shows increasingly larger variability with increasing system sizes.
A more thorough examination of the data shows widely varying performance across individual subquorums in the larger configurations.
After instrumenting the experiments to diagnose the problem, we determined it was a bug in the networking code, which we repaired and improved.
By aggregating append entries messages from clients while consensus messages were in-flight, we managed to dramatically increase the performance of single quorums and reduce the number of messages sent.
This change also had the effect of ensuring that the variability was decreased in our final experiment.

\begin{figure}
    \begin{center}
        \includegraphics[width=5in]{figures/ch03_ec2_latency_cumfreq.pdf}
    \end{center}
    \renewcommand{\baselinestretch}{1}
    \small\normalsize

    \begin{quote}
        \caption[HC Cumulative Latency Distribution]{}
        \label{fig:ch03_ec2_latency_cumfreq}
    \end{quote}
\end{figure}
\renewcommand{\baselinestretch}{2}
\small\normalsize

The effect of saturation is also demonstrated in Figure~\ref{fig:ch03_ec2_latency_cumfreq}, which shows cumulative latency distributions for different system sizes holding the  workload (number of concurrent clients) constant.
The fastest (24/3) shows nearly 80\% of client write requests being serviced in under 2 msec.
Larger system sizes are faster because the smaller systems suffer from contention (25 clients can saturate a single subquorum).
Because throughput is directly related to commit latency, throughput variability can be mitigated by adding additional subquorums to balance load.

Besides pure performance and scaling, HC is also motivated by the need to adapt to varying environmental conditions.
In the next set of experiments, we explore two common runtime scenarios that motivate adaptation: shifting client workloads and failures.
We show that HC is able to adapt and recover with little loss in performance. These scenarios are shown in Figures~\ref{fig:ch03_umd_sawtooth} and \ref{fig:ch03_umd_fault_tolerance} as throughput over time, where vertical dotted lines indicate an epoch change.

\begin{figure}
    \begin{center}
        \includegraphics[width=5in]{figures/ch03_umd_sawtooth.pdf}
    \end{center}
    \renewcommand{\baselinestretch}{1}
    \small\normalsize

    \begin{quote}
        \caption[Sawtooth Graph]{9/3 system adapting to changing client access patterns by repartitioning the tag space so that clients are co-located with subquorums that serve tags they need.}
        \label{fig:ch03_umd_sawtooth}
    \end{quote}
\end{figure}
\renewcommand{\baselinestretch}{2}
\small\normalsize

The first scenario, described by the time series in Figure~\ref{fig:ch03_umd_sawtooth} shows an HC 3-replica configuration moving through two epoch changes.
Each epoch change is triggered by the need to localize tags accessed by
clients to nearby subquorums.
% The experiment was run over machines with widely varying latency.
The scenario shown starts with all clients co-located with the subquorum serving the tag they are accessing.
However, clients incrementally change their access patterns first to a tag located on one remote subquorum, and then to the tag owned by the other.
In both cases, the root quorum adapts the system by repartitioning the tagspace such that the tag defining their current focus is served by the co-located subquorum.

\begin{figure}
    \begin{center}
        \includegraphics[width=5in]{figures/ch03_umd_fault_tolerance.pdf}
    \end{center}
    \renewcommand{\baselinestretch}{1}
    \small\normalsize

    \begin{quote}
        \caption[HC Fault Repartitioning]{9/3 System that adapts to failure (partition) of entire subquorum. After timeout, the root quorum re-partitions the tag allocated to the failed subquorum among the other two subquorums.}
        \label{fig:ch03_umd_fault_tolerance}
    \end{quote}
\end{figure}
\renewcommand{\baselinestretch}{2}
\small\normalsize

Figure~\ref{fig:ch03_umd_fault_tolerance} shows a 3-subquorum configuration where one entire subquorum becomes partitioned from the others.
After a timeout, the root uses an epoch change to re-allocate the tag of the partitioned subquorum over the two remaining subquorums.
The partitioned subquorum eventually has an \emph{obligation timeout}, after which the root quorum is not obliged to leave the tag with the current subquorum.
The tag may then be re-assigned to any other subquorum.
Timeouts are structured such that by the time an obligation timeout fires, the root quorum has already re-mapped that subquorum's tag to other subquorums.
As a result, the system is able to recover from the partition as fast as possible.
In this figure, the repartition occurs through two epoch changes, the first allocating part of the tagspace to the first subquorum, and the second allocating the rest of the tag to the other.
Gaps in the graph are periods where the subquorums are electing local leaders.
This may be optimized by having leadership assigned or maintained through root consensus.

\begin{figure}
    \begin{center}
        \includegraphics[width=5in]{figures/ch03_geoconsensus_total_throughput.pdf}
    \end{center}
    \renewcommand{\baselinestretch}{1}
    \small\normalsize

    \begin{quote}
        \caption[Planetary Scale HC Throughput]{Throughput of the final implementation of Alia across 105 replicas in 15 AWS regions with size 3 subquorums colocated in each region. HC linearly scales as the number of replicas increases, also increasing the fault tolerance and the durability of the system.}
        \label{fig:ch03_geoconsensus_total_throughput}
    \end{quote}
\end{figure}
\renewcommand{\baselinestretch}{2}
\small\normalsize

In our final implementation we ran our repaired version of HC at a planetary scale.
We created a system with 105 replicas in 15 regions in 5 continents.
The system allocated size 3 subquorums round-robin to each region such that the largest system was comprised of 6 subquorums per region with 1 hot-spare per region.
Figure~\ref{fig:ch03_geoconsensus_total_throughput} shows the global blast throughput of the system, the sum of throughput of client process that fired off 1000 concurrent requests, timing the complete response.
To mitigate the effect of global latency, each region ran independent blast clients to its local subquorums, forwarding to remote quorums where necessary.
To ensure that the system was fully throttled during the throughput experiment, we timed the clients to execute simultaneously using the AWS Time Sync service to ensure that clocks were within 100 nanoseconds of each other.
In these results we show that our HC implementation does indeed scale linearly.
Adding more nodes to the system increases the fault tolerance (e.g. by allocating hot spares) if enough nodes are added to add another subquorum, the capacity of the system to handle client requests is also increased.

\section{Conclusion}
\label{ch03_conclusion}

Most consensus algorithms have their roots in the Paxos algorithm, originally described in parliamentary terms.
The metaphor of government still applies well as we look at the evolution of distributed coordination as systems have grown to include large numbers of processes and geographies.
Systems that use a dedicated leader are easy to reason about and implement. However, as in chess, if the leader fails the system cannot make any progress.
Simple democracies for small groups solve this problem but do not scale, and as the system grows, it fragments into tribes.
Inspired by modern governments, we have proposed a representative system of consensus, hierarchical consensus, such that replicas elect leaders to participate in a root quorum that makes decisions about the global state of the system.
Local decision making, the kind that effects only a subset of clients and objects is handled locally by subquorums as efficiently as possible.
The result is a a hierarchy of decision making that takes advantage of hierarchies that already exist in applications.

Hierarchical Consensus is an implementation and extension of Vertical Paxos.
Like Vertical Paxos, HC reasons about consistency across all objects by identifying commands with a grid ordering (rather than a log ordering) and is reconfigurable to adapt to dynamic environments that exist in geo-replicated systems.
Adaptability allows HC to exploit locality of access, allowing for high performance coordination, even with replication across the wide area.
HC extends Vertical Paxos to ensure that intersections exist between the subquorums and the root quorum, to guarantee operations between subquorums, and to ensure that the system operates as a coordinated whole.
To scale the consensus protocol of the root quorum, we propose a novel approach, delegation, to ensure that all replicas participate in consensus but limit the number and frequency of messages required to achieve majority.
Finally, we generalized HC from primary-backup replication to describe more general online replication required by distributed databases and file systems.

In the next chapter we will explore a hybrid consistency model implemented by federating replicas that participate in different consistency protocols.
In a planetary scale network, HC provides the strong consistency backbone of the federated model, increasing the overall consistency of the system by making coordinating decisions at a high level, and allowing high availability replicas in the fog operate independently where necessary.
