%Chapter 2

\renewcommand{\thechapter}{2}

\chapter{Challenges and Motivations}

Is a planetary-scale data system really necessary?

What challenges to these systems face?

What does consistency mean in this model?


\section{Building Geo-Replicated Services: Case Studies}

What they didn't do and what we do better.

BigTable \& Spanner

S3

Aurora \& Cockroach DB

Approaches: sharding, independent objects, buckets, slow reads

\section{Requirements for Data Systems}

Failure is common
    Disk failure/replica failure (ODS)
    Network failure (when repaired, replica comes back online)
    Unreliability: messages have highly variable latency, out of order messaging
    Partitions, part of the system cannot speak to the rest of the system

In geo-systems large latency is not the issue!
    There is a physical limit to message traffic
    Writes must be applied in order, reads can reason about staleness
    Access patterns are also location-dependent

Durability
    Normally 3 disk replication ensures 2 failures
    We need to ensure zone+1 disk failures (re Aurora)
    User-specific data should be accessible everywhere

Fault-Tolerance \& availability
    System should still be available if nodes fail
    Should be available if zones fail (e.g. hurricane or disaster) at higher latency
    System should be available at lower consistency if even one replica is available

Adaptability
    Should respond to changes in user access patterns
    Should be able to add and remove nodes from the system
    Should scale with more regions and more replicas

\section{System Architecture}

Describe the architecture of tier 1: HC, tier 2: Federated Fog

Describe the consistency guarantees we claim

Base application is a key/value store

File-system is built upon the key/value store as an object FS
