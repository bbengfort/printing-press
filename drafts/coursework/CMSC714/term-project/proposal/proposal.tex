\documentclass[11pt,letterpaper]{article}

\usepackage{graphicx}
\usepackage[margin=1in]{geometry}
\usepackage[T1]{fontenc}
\usepackage[utf8]{inputenc}
\usepackage{authblk}
\usepackage{fancyhdr}
\usepackage{lastpage}
\usepackage[parfill]{parskip}
\usepackage{subcaption}

\pagestyle{fancyplain}
\fancyhf{}
\fancyfoot[R]{\footnotesize Page \thepage\ of \pageref{LastPage}}

\renewcommand{\headrulewidth}{0.0pt} % No header rule
\renewcommand{\footrulewidth}{0.4pt} % Thin footer rule

\begin{document}

\title{Streaming Project Proposal}
\author[1]{Benjamin Bengfort}
\author[2]{Allen Leis}
\author[1]{Konstantinos Xirogiannopoulos}
\affil[ ]{Department of Computer Science}
\affil[ ]{University of Maryland}
\affil[1]{\textit{\{bengfort,kostasx\}@cs.umd.edu}}
\affil[2]{\textit{aleis@umd.edu}}

\date{October 12, 2015}

\maketitle
\section{Introduction}
Some introduction talking about how cool and important streaming is, its applications etc.

\section{Proposals}
Below, we enumerate a few kernels of ideas that may be interesting to pursue as research projects for the purposes of the course at first, and then to push forward towards a potential publication.

\begin{enumerate}
  \item \textbf{Accelerated Streaming via Communication Between Streams}
   Streaming algorithms traditionally incorporate discrete streams of data that flow through layers of computation. If we were to assemble an appropriate set of streaming algorithms and try to reason about ways that communication between the streams would be beneficial. This would involve coming up with novel ideas and techniques for accellerating computation and doing experiments to see what the trade-offs and differences would be. Our hope is that while studying these algorithms, we will be able to come up with ways portions of the algorithms could be accelerated and be able to implement and evaluate those ideas. 
\end{enumerate}

\bibliographystyle{plain}
\bibliography{papers}

\end{document}
