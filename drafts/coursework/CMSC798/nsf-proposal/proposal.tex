% TEMPLATE for Usenix papers, specifically to meet requirements of
%  USENIX '05
% originally a template for producing IEEE-format articles using LaTeX.
%   written by Matthew Ward, CS Department, Worcester Polytechnic Institute.
% adapted by David Beazley for his excellent SWIG paper in Proceedings,
%   Tcl 96
% turned into a smartass generic template by De Clarke, with thanks to
%   both the above pioneers
% use at your own risk.  Complaints to /dev/null.
% make it two column with no page numbering, default is 10 point

% Munged by Fred Douglis <douglis@research.att.com> 10/97 to separate
% the .sty file from the LaTeX source template, so that people can
% more easily include the .sty file into an existing document.  Also
% changed to more closely follow the style guidelines as represented
% by the Word sample file. 

% Note that since 2010, USENIX does not require endnotes. If you want
% foot of page notes, don't include the endnotes package in the 
% usepackage command, below.

% This version uses the latex2e styles, not the very ancient 2.09 stuff.
\documentclass[letterpaper,twocolumn,10pt]{article}
\usepackage{usenix,epsfig,endnotes}
\begin{document}

%don't want date printed
\date{}

%make title bold and 14 pt font (Latex default is non-bold, 16 pt)
\title{\Large \bf Evolving the Internet for Scalability and Security with
    Content-Centric Networking}

%for single author (just remove % characters)
\author{
{\rm B. Bengfort, S.\ Herwig, A.\ Mathur, M.\ Shao}\\
University of Maryland, College Park
} % end author

\maketitle

% Use the following at camera-ready time to suppress page numbers.
% Comment it out when you first submit the paper for review.
\thispagestyle{empty}

\section{Objectives}
%objectives and scientific, engineering, or educational significance of the%
%proposed work.  Present the merits of the the proposed project clearly.
The Internet has become a platform for content dissemination.  Unfortunately,
the underlying host-centric, conversation-model of TCP/IP is inefficient
for massive content distribution.  New information domains, such as mobile
technologies and the Internet of the Things, challenge the current assumptions
for constant connectivity, while perennial data breaches highlight the
insufficiency of current privacy and security solutions.

This project proposes extending the research of the Named-Data Networking (NDN)
project for achieving scalable and secure communications based on a
content-centric networking paradigm.  In particular, the project will
explore models for distributed, decentralized, trust for the purposes of
signing and verifying named data.

\section*{Methods}
%suitability of the methods to be employed.
In contrast to TCP/IP, which places responsibilty for security on the
endpoints, NDN secures content by requiring data producers to cryptographically
sign every data packet.  The publisher's signature ensures integrity and
enables determination of data provenance.  Moreover, clients assess the trust
of a data packet independent of where or how the data was obtained.

A critical problem is therefore how the client verifies the authenticity of a
publisher's public key.  In contrast to the inflexible PKI scheme that
dominates the current Internet, this project seeks to extend the concepts of a
web of trust to NDN, thereby enabling secure communication without requiring
pre-agreed trust anchors.  The project will also explore concepts of
reputation-based trust, as a means of providing redundancy of verification, and
allowing data and key names to be independent.

\section*{Qualifications}
%qualifications of the investigator and the grantee organization.
The submitters of this proposal are the chief investigators.  They are computer
science PhD students at the University of Maryland with a focus on networking
and systems. 

\section*{Intellectual Merit}
The project will explore fundamental issues in communication theory, as well as
theoretical research in the evaluation of trust.  

In particular, the project will explore prior work in constructing an algebra
for describing trust and measuring the propagation of trust throughout a
system.  A key element in such an algebra is an allowance for the temporal
nature of trust, and the need for expiration and revocation.  Moreover, the
alegebra must allow for the validation of authenticity of a new, untrusted, party
that is introduced into a system.

\section*{Broader Impact}
%effect of the activity on the infrastructure of science, engineering, and
%education.
The project proposes research that supports evolving the Internet to better
scale for secure content distribution.  Such research could be influential in
establishing a future Internet architecture that is more appropriate to current
and emerging use-cases.  As the existing Internet has caused wide-scale
societal changes, a component of this project is assessing the societal impacts
of any proposed changes to the current Internet.

\section*{Funding}
%amount of funding required.
The projects requests 300K a year for three years.  The funds will support the
researchers, allow for the purchase of networking and server equipment, and
provide for the sharing of research at yearly academic and industrial
conferences.

\end{document}


