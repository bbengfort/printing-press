\documentclass[11pt,letterpaper]{article}

\usepackage{graphicx}
\usepackage[margin=1in]{geometry}
\usepackage{amsmath}
\usepackage[T1]{fontenc}
\usepackage[utf8]{inputenc}
\usepackage{authblk}
\usepackage{fancyhdr}
\usepackage{lastpage}
\usepackage[parfill]{parskip}

\pagestyle{fancyplain}
\fancyhf{}
\fancyfoot[R]{\footnotesize Page \thepage\ of \pageref{LastPage}}

\renewcommand{\headrulewidth}{0.0pt} % No header rule
\renewcommand{\footrulewidth}{0.4pt} % Thin footer rule

\begin{document}

\title{Visual Discovery of Communication Patterns in Email Networks}

\author[ ]{Benjamin Bengfort}
\author[ ]{Konstantinos Xirogiannopoulos}
\affil[ ]{Department of Computer Science}
\affil[ ]{University of Maryland}
\affil[ ]{\textit{\{bengfort,kostasx\}@cs.umd.edu}}

\date{April 6, 2015}

\maketitle

\section*{Introduction}

Use Gephi \cite{gephi_gephi-open_2010} to visualize the network from a GraphML \cite{brandes_graph_2010} file. Gephi is used for visual exploration and mapping of networks \cite{bastian_gephi:_2009} and can do statistical analysis of small to medium networks \cite{mcsweeney_gephi_2009}.

Force Atlas 2 Continuous Layout \cite{jacomy_forceatlas2_2014}

Louvian Community Detection \cite{de_meo_generalized_2011}

Edge Bundling \cite{pupyrev_edge_2012}

Dynamic network connections within Twitter conversations are explored in \cite{bruns_how_2012}.

\subsection*{Using Email as a Dataset}

\subsection*{Data Wrangling for Gephi}

\section*{Insights}

In this section we describe the exploration of the email graphs of the authors - both independently and joined using Gephi.

\subsection*{Headline 1}

Explore motif simplification (complexity reduction) using edge bundling

\subsection*{Headline 2}

Explore centrality measures


\subsection*{Headline 3}

Explore clusters and communities

\section*{Critique}

\section*{Conclusion}

\bibliographystyle{plain}
\bibliography{paper}

\end{document}
