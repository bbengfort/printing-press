% Glossary Definitions
% Use the following to define a glossary entry:
%
% \newglossaryentry{label}
% {
%   name=name,
%   description={description},
%   sort=key,
%   plural=pluralized,
%   symbol=symbol
% }
%
% Define acronyms as follows:
%
% \newacronym{<label>}{<abbrv>}{<full>}
%
% For more see: https://en.wikibooks.org/wiki/LaTeX/Glossary#Defining_terms

\newglossaryentry{linearizability}
{
    name=linearizability
}

\newglossaryentry{eventual}
{
    name={eventual consistency},
    description={if no new updates are made to an object, eventually all accesses will return the last updated value}
}

\newglossaryentry{causal}
{
    name={causal consistency},
    description={objects that are causally related be ordered with respect to a read but still relaxes full linearizability by allowing partial ordering of unrelated objects}
}

 \newglossaryentry{crdt}
{
  name={CRDT}
  description={In distributed computing, a conflict-free replicated data type (abbreviated CRDT) is a type of specially-designed data structure used to achieve strong eventual consistency (SEC) and monotonicity (absence of rollbacks). As their name indicates, a CRDT instance is distributed into several replicas; each replica can be mutated promptly and concurrently; the potential divergence between replicas is however guaranteed to be eventually reconciled through downstream synchronisation (off the critical path); consequently CRDTs are known to be highly available. \cite{crdt_wikipedia} }
}
