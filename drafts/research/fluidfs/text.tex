% see https://www.usenix.org/sites/default/files/template.la_.txt for original.
\documentclass[letterpaper,twocolumn,10pt]{article}
\usepackage{usenix,epsfig}
\begin{document}

%don't want date printed
\date{}

%make title bold and 14 pt font (Latex default is non-bold, 16 pt)
\title{\Large \bf Strong Consistency in a Geo-Replicated File System}

%for single author (just remove % characters)
\author{
{\rm Benjamin Bengfort}\\
University of Maryland\\
bengfort@cs.umd.edu
\and
{\rm Pete Keleher}\\
University of Maryland\\
keleher@cs.umd.edu
} % end author

\maketitle

% Use the following at camera-ready time to suppress page numbers.
% Comment it out when you first submit the paper for review.
% \thispagestyle{empty}



\section*{Implementation}

The FluidFS File System is implemented in Golang as a filesystem in user
space using the \texttt{bazil.org/fuse} implementation in pure Go
(without libfuse bindings).
This dependency is a custom implementation of the kernel-userspace
communication protocol inspired by the FUSE library.

FluidFS mounts one or more mount points defined in a host-specific,
\texttt{fstab}-like configuration file.
Each mount point constructs an independent file system from the perspective
of the kernel, whose top level is the root directory where the FS is mounted.
FluidFS, however treats each mount point as a subdirectory of an abstract
global file system, specified by a unique \textit{prefix} or TLD (top level
directory) that must also be identified in the configuration file.
When two mount points are specified with the same prefix (either locally or
on different hosts), FluidFS treats each mount point as a partial replica of
the global file system.

FluidFS is a virtual distributed file system that does not manage a disk
directly but instead primarily manages objects (files).
A file is a collection of version metadata and binary blobs that are stored
separately. A \textit{file version} is represented by a single piece of meta
data that contains two Lamport scalar compound numbers: the version number and the parent of the current version.


{\footnotesize \bibliographystyle{acm}
\bibliography{references}}

\end{document}
