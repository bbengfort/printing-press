%%%% Proceedings format for most of ACM conferences (with the exceptions listed below) and all ICPS volumes.
\documentclass[sigconf]{acmart}
%%%% As of March 2017, [siggraph] is no longer used. Please use sigconf (above) for SIGGRAPH conferences.

%%%% Proceedings format for SIGPLAN conferences
% \documentclass[sigplan, anonymous, review]{acmart}

%%%% Proceedings format for SIGCHI conferences
% \documentclass[sigchi, review]{acmart}

%%%% To use the SIGCHI extended abstract template, please visit
% https://www.overleaf.com/read/zzzfqvkmrfzn


\usepackage{booktabs} % For formal tables


% Copyright
%\setcopyright{none}
%\setcopyright{acmcopyright}
%\setcopyright{acmlicensed}
\setcopyright{rightsretained}
%\setcopyright{usgov}
%\setcopyright{usgovmixed}
%\setcopyright{cagov}
%\setcopyright{cagovmixed}


% DOI
% \acmDOI{10.475/123_4}

% ISBN
% \acmISBN{123-4567-24-567/08/06}

%Conference
\acmConference[PaPoC'18]{5th Workshop on Principles and Practice of Consistency for Distributed Data}{April 2018}{Porto, Portugal}
\acmYear{2018}
\copyrightyear{2018}


% \acmArticle{4}
% \acmPrice{15.00}

% These commands are optional
%\acmBooktitle{Transactions of the ACM Woodstock conference}
% \editor{Jennifer B. Sartor}
% \editor{Theo D'Hondt}
% \editor{Wolfgang De Meuter}


\begin{document}
\title{Improved Anti-Entropy with Reinforcement Learning}
% \titlenote{Produces the permission block, and copyright information}
% \subtitle{Extended Abstract}
% \subtitlenote{The full version of the author's guide is available as
%   \texttt{acmart.pdf} document}


\author{Benjamin Bengfort}
% \authornote{}
% \orcid{}
\affiliation{%
  \institution{University of Maryland}
% \streetaddress{}
% \city{College Park}
% \state{MD}
% \postcode{20742}
}
\email{bengfort@cs.umd.edu}

\author{Pete Keleher}
% \authornote{}
% \orcid{}
\affiliation{%
  \institution{University of Maryland}
% \streetaddress{}
% \city{College Park}
% \state{MD}
% \postcode{20742}
}
\email{keleher@cs.umd.edu}


\begin{abstract}
    Groups of strongly consistent devices can efficiently replicate data under
ideal (data center) conditions, but become less effective in dynamic and
heterogeneous environments.  Eventually consistent devices efficiently
tolerate both faults and dynamic conditions but are slow to converge on a
single ordering of system events.

We propose ``federated consistency'', an approach that combines the strengths
of both approaches into a single protocol. Federated groups use a strongly
consistent inner core of devices to maintain a totally ordered, fault-tolerant
sequence of events.  A cloud of eventually-consistent devices disseminates
orderings and enables progress despite varying connectivity and partitions.
We use a detailed event simulation to show that a group of federated devices
can obtain the key advantages of both approaches. 

\end{abstract}

%
% The code below should be generated by the tool at
% http://dl.acm.org/ccs.cfm
% Please copy and paste the code instead of the example below.
%
\begin{CCSXML}
<ccs2012>
<concept>
<concept_id>10010147.10010257.10010258.10010261</concept_id>
<concept_desc>Computing methodologies~Reinforcement learning</concept_desc>
<concept_significance>500</concept_significance>
</concept>
<concept>
<concept_id>10010520.10010575.10011743</concept_id>
<concept_desc>Computer systems organization~Fault-tolerant network topologies</concept_desc>
<concept_significance>500</concept_significance>
</concept>
<concept>
<concept_id>10010520.10010575.10010577</concept_id>
<concept_desc>Computer systems organization~Reliability</concept_desc>
<concept_significance>300</concept_significance>
</concept>
<concept>
<concept_id>10010520.10010575.10010578</concept_id>
<concept_desc>Computer systems organization~Availability</concept_desc>
<concept_significance>300</concept_significance>
</concept>
</ccs2012>
\end{CCSXML}

\ccsdesc[500]{Computing methodologies~Reinforcement learning}
\ccsdesc[500]{Computer systems organization~Fault-tolerant network topologies}
\ccsdesc[300]{Computer systems organization~Reliability}
\ccsdesc[300]{Computer systems organization~Availability}


\keywords{Eventual Consistency, Anti-Entropy, Reinforcement Learning}


\maketitle

% see https://www.usenix.org/sites/default/files/template.la_.txt for original.
\documentclass[letterpaper,twocolumn,10pt]{article}
\usepackage{usenix,epsfig}
\begin{document}

%don't want date printed
\date{}

%make title bold and 14 pt font (Latex default is non-bold, 16 pt)
\title{\Large \bf Strong Consistency in a Geo-Replicated File System}

%for single author (just remove % characters)
\author{
{\rm Benjamin Bengfort}\\
University of Maryland\\
bengfort@cs.umd.edu
\and
{\rm Pete Keleher}\\
University of Maryland\\
keleher@cs.umd.edu
} % end author

\maketitle

% Use the following at camera-ready time to suppress page numbers.
% Comment it out when you first submit the paper for review.
% \thispagestyle{empty}



\section*{Implementation}

The FluidFS File System is implemented in Golang as a filesystem in user
space using the \texttt{bazil.org/fuse} implementation in pure Go
(without libfuse bindings).
This dependency is a custom implementation of the kernel-userspace
communication protocol inspired by the FUSE library.

FluidFS mounts one or more mount points defined in a host-specific,
\texttt{fstab}-like configuration file.
Each mount point constructs an independent file system from the perspective
of the kernel, whose top level is the root directory where the FS is mounted.
FluidFS, however treats each mount point as a subdirectory of an abstract
global file system, specified by a unique \textit{prefix} or TLD (top level
directory) that must also be identified in the configuration file.
When two mount points are specified with the same prefix (either locally or
on different hosts), FluidFS treats each mount point as a partial replica of
the global file system.

FluidFS is a virtual distributed file system that does not manage a disk
directly but instead primarily manages objects (files).
A file is a collection of version metadata and binary blobs that are stored
separately. A \textit{file version} is represented by a single piece of meta
data that contains two Lamport scalar compound numbers: the version number and the parent of the current version.


{\footnotesize \bibliographystyle{acm}
\bibliography{references}}

\end{document}


\bibliographystyle{ACM-Reference-Format}
\bibliography{papers}

\end{document}
