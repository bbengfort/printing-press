\documentclass[11pt,letterpaper]{article}
\usepackage{cogsys}
\usepackage{cogsysapa}
\usepackage[utf8]{inputenc}
\usepackage[T1]{fontenc}
\usepackage{times}
\usepackage[pdftex]{graphicx} % use this when importing PDF files

 % First page headings for accepted submissions.
\cogsysheading{X}{20XX}{1-6}{3/2015}{X/20XX}
 % First page headings for poster submissions.
%\cogsysposterheading{First}{2012}{1-18}

\ShortHeadings{Interactive Reasoning to Solve Knowledge Goals}
              {B.\ Bengfort and M.\ Cox}

\begin{document}

\title{Interactive Knowledge-Goal Reasoning}

\author{Benjamin Bengfort}{bengfort@cs.umd.edu}
\address{Department of Computer Science, University of Maryland,
         College Park, MD 20742 USA}
\author{Michael Cox}{michael.cox@wright.edu}
\address{Wright State Research Institute,
         Beavercreek, OH 45431 USA}
\vskip 0.2in

% Title (1000 readers)
% Abstract (4 sentences, 100 readers)
% Introduction (1 page, 100 readers)
% The problem (1 page, 10 readers)
% My idea (2 pages, 10 readers)
% The details (5 pages, 3 readers)
% Related work (1-2 pages, 10 readers)
% Conclusions and further work (0.5 pages)

\begin{abstract}

\end{abstract}

\section{Introduction}

\section{Knowledge Goals}

\subsection{Components of a Question}

\subsection{A Taxonomy of Questions}

\subsection{Goal Trajectories}

\section{Case Studies}

\subsection{Conceirge}

\subsection{}

\subsection{}

\section{Related Work}

\section{Conclusion}

% \newpage

\begin{acknowledgements}
\noindent
% Please place your acknowledgements in an unnumbered section at the
% end of the paper. Typically, this will include thanks to reviewers
% who gave useful comments, to colleagues who contributed to the ideas,
% and to funding agencies or corporate sponsors that provided financial
% support.
\end{acknowledgements}

\vspace{-0.25in}

{\parindent -10pt\leftskip 10pt\noindent
\bibliographystyle{cogsysapa}
\bibliography{kgworkshop}

}

% Leave a blank line before the closing brace to ensure the final
% reference has the proper indentation.

\end{document}
